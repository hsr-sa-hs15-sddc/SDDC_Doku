\chapter{Persönliche Berichte}

\section{Silvan Adrian}


Zu beginn sah für mich noch alles super aus, alleine schon weil das Thema SDDC für 
uns das einzige das uns wirklich zusagte und auf dem Gebiet immer noch ziemlich 
grosser Bedarf dafür besteht und ausser OpenStack und CloudStack gibt es nicht 
viele Anbieter, die einem gerade in den Sinn kommen.
Als die Arbeit dann schliesslich anfing, ging es erst Mal darum alle vorhanden 
Anbieter, Libraries und Softwares anzuschauen und dadurch das es hier ein sehr 
grosses Angebot besteht dauerte die Analyse schlussendlich auch lange.
Zu Anfang suchten wir eher nach Public Cloud Anbieter Libaries und Anbieter, bis 
wir uns schliesslich entschlossen haben Private ``Cloud'' Anbieter zu 
unterstützen, dabei sind wir auf Libvirt gestossen, welche nun die Library ist 
welche wir verwendet haben.
Danach hiess es noch das ganze Programmiertechnisch umzusetzen und die 
Möglichkeit offen zu lassen für weitere Libaries.
Zu Anfang brauchten wir plain Java und wollten dann mit einzelnen Komponenten für die 
RESTful API und Datenbank aufbauen.
Nach dem 1. Sprint wechselten wir jedoch auf ein Java Framework, welches einiges 
vereinfachte, so konnten wir alles miteinander ausliefern und aufschalten.
Danach bauten wir noch so viel wie möglich ein, jedoch fehlte schlichtweg die 
Zeit noch eine bessere Testumgebung aufzusetzen, um unsere Software mit mehr 
Anbietern als nur mit KVM zu testen, aber die Möglichkeiten sind gegeben und 
müssten nur noch getestet werden.

Im Nachhinein bin ich mit unserem Ergebnis zufrieden, wir haben das Beste 
rausgeholt was möglich war um eine Software zur Verfügung zu stellen, welche es 
im einfachen Rahmen erlaubt Services zu abonnieren und zu kündigen.
Jetzt müssten nur mal noch weiter Libraries eingebunden werden und eine bessere 
Lösung für die Configfiles umgesetzt werden, damit unter Servicemodulen eine Art 
Abhängigkeit besteht um Infos von anderen erstellten Servicemodulen auslesen zu 
können.


\section{Fabian Binna}