\chapter{Persönliche Berichte}

\section{Silvan Adrian}

SDDC war für mich das einzige Thema, dass mir wirklich zusagte und es 
schlichtweg die Zukunft für modernere Data Centers ist.
Als die Arbeit dann schliesslich anfing, ging es erst Mal darum alle vorhanden 
Anbieter, Libraries und Softwares zu analysieren. Dadurch das ein sehr 
grosses Angebot besteht dauerte die Analyse auch lange.
Zu Anfang suchten wir nach Public Cloud Anbieter Libraries, bis 
wir uns schliesslich entschlossen haben Private ``Cloud'' Anbieter zu 
unterstützen. Dabei sind wir auf Libvirt gestossen, welche die 
Library ist mit der wir die Arbeit abgeschlossen haben.
Danach hiess es noch das ganze Programmiertechnisch umzusetzen und die 
Möglichkeit offen zu lassen für weitere Libraries.
Zu Anfang brauchten wir plain Java nach dem 1. Sprint wechselten wir jedoch auf ein 
Java Framework, welches einiges 
vereinfachte, so konnten wir alles miteinander ausliefern und aufschalten.
Danach bauten wir noch so viel wie möglich ein und erstellten dazu ein Dashboard,
 jedoch fehlte schlichtweg die 
Zeit noch eine bessere Testumgebung aufzusetzen, um unsere Software mit mehr 
Anbietern als nur mit KVM zu testen, aber die Möglichkeiten sind gegeben und 
müssten nur noch getestet werden.
\newline
Im Nachhinein bin ich mit unserem Ergebnis zufrieden, wir haben das Beste 
rausgeholt was möglich war in der kurzen Zeit.
Nun existiert eine Software, welche es einem im einfachen Rahmen erlaubt 
Services/Servicemodule zu abonnieren, zu kündigen und zu verwalten.
Jetzt müssten nur mal noch weiter Libraries eingebunden werden und eine bessere 
Lösung für die Configfiles umgesetzt werden, dann kann auch etwas daraus entstehen.

\newpage
\section{Fabian Binna}

Software Defined Data Center ist die Zukunft. Viele Unternehmen wollen die Intelligenz eines Rechenzentrums zentralisieren.
 Das ermöglicht einen sehr agilen Betrieb, bei dem sowohl vertikal als auch horizontal skaliert werden kann. 
 Für mich war SDDC das einzig interessante Projekt, das zur Auswahl stand. Wir wollten uns einer neuen
  Herausforderung stellen und nicht straight-forward einen Webservice implementieren.\\
\newline
Da ich auf diesem Gebiet nicht besonders bewandert war, musste ich zuerst einiges lernen. Die Begriffe und Konzepte waren mir aber relativ schnell geläufig, so dass ich in Diskussionen meine Ideen gut einbringen konnte. Wir haben einige Produkte analysiert und versucht die Konzepte und Technologien auf eine abstrakte Ebene zu bringen. Während den Meetings wurde oft darüber diskutiert was die Software am Ende können muss und vor allem wo die Abstraktion ansetzt. Das führte dazu das sich die Architektur und die Libraries zweimal komplett änderten. Die Besprechungen und Änderungen waren essentiell um einen neuen Ansatz zu verfolgen. Das Resultat ist keine Kopie einer anderen Software sonder ein Proof-of-Concept für eine generische und erweiterbare Abhandlung von Service Modulen, die zusammen einen Service bilden.\\
\newline
Wir haben viele neue Technologien kennen gelernt. Der Spring Framework in Verbindung mit anderer Software erleichterte die Entwicklung des Projekts enorm. Schlussendlich bin ich mit dem Resultat des Projekts zufrieden. Der Anfang war nicht leicht und es ist schade, dass einige unserer Ideen nicht verwirklicht werden konnten, aber ich sehe Potenzial aus diesen Konzepten und Ideen etwas Grösseres zu entwickeln.