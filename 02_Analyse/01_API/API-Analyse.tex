\documentclass[11pt]{scrartcl}

\title{API Analyse}
\author{Silvan Adrian \\ Fabian Binna}
\date{\today{}}

\usepackage[ngerman]{babel}
\usepackage[automark]{scrpage2}
\usepackage{hyperref}
\usepackage{color}
\usepackage[normalem]{ulem}
\usepackage{scrpage2}
\usepackage{graphicx}
\usepackage{tabularx}
\graphicspath{ {../22_Grafiken/01_Logo/}{images/}{../../22_Grafiken/01_Logo/} }
\pagestyle{scrheadings}

\clearscrheadfoot
\ihead{\includegraphics[scale=0.3]{SDDC}}
\ohead{Projekt: SDDC}
\ifoot{API Analyse}
\cfoot{Version: 1.00}
\ofoot{Datum: \today{}}
\setheadsepline{0.5pt}
\setfootsepline{0.5pt}

\usepackage{ucs}
\usepackage[utf8]{inputenc}
\usepackage[T1]{fontenc}


\begin{document}
\def\arraystretch{1.5}
\begin{titlepage}
\begin{center}
\vspace{10em}
\includegraphics[scale=2]{SDDC}
\vspace{10em}
\end{center}
\begin{center}
\huge {API Analyse}
\end{center}
\begin{center}
\vspace{10em}
\LARGE {Silvan Adrian} \\
\LARGE {Fabian Binna}
\end{center}

\end{titlepage}

\newpage
\section{Änderungshistorie}
\begin{tabularx}{\linewidth}{l l l l}
\textbf{Datum} & \textbf{Version} & \textbf{Änderung}  & \textbf{Autor} \\
\hline
\textbf{17.09.15} & 1.00 & Erstellung des Dokuments & Gruppe \\

\end{tabularx}

\newpage
\tableofcontents
\newpage

\section{APIs}
\subsection{\href{https://libcloud.apache.org}{Libcloud}}
\textbf{Sprache: }Python\\
\textbf{Wichtigste Provider: }Rackspace, Amazon web services, CloudStack, OpenStack, DigitalOcean, Eucalyptus, Joyent, Linode, exoscale,NephoScale, Google Cloud Platform, Zerigo, CloudSigma, iKoula, libvirt\\

\subsection{\href{https://jclouds.apache.org}{jClouds}}
\textbf{Sprache: }Java\\
\textbf{Wichtigste Provider: }OpenStack, Docker, DigitalOcean, Google Cloud Platform, Rackspace, HP Cloud, CloudStack, Amazon web services, abiquo, CloudSigma, joyent\\

\subsection{\href{https://github.com/esl/elibcloud}{elibcloud}}
\textbf{Sprache: }Erlang\\
elibcloud ist ein Wrapper für libcloud.\\

\subsection{\href{https://github.com/fog/fog/blob/master/lib/fog/openstack/docs/getting_started.md}{fog}}
\textbf{Sprache: }Ruby\\
\textbf{Wichtigste Provider: }CloudSigma, CloudStack, GoGrid, Google Cloud Platform, Joyent, Libvirt, Linode, OpenStack, OpenVZ, Rackspace, Zerigo, IBM, HP\\

\subsection{\href{https://github.com/pkgcloud/pkgcloud}{pkgcloud}}
\textbf{Sprache: }JavaScript (Node.js)\\
\textbf{Wichtigste Provider: }Amazon, Azure, DigitalOcean, Joyent, OpenStack, Rackspace, Google, HP,\\

\newpage

\section{Support}
\subsection{Compute}
\subsection{Storage}
\subsection{Network}
\subsection{Other}
Database, DNS, Block Storage, Load Balancers, Orchestration\\
\end{document}










