\documentclass[11pt]{scrartcl}

\title{Anforderungen Analyse}
\author{Silvan Adrian \\ Fabian Binna}
\date{\today{}}

\usepackage[ngerman]{babel}
\usepackage[automark]{scrpage2}
\usepackage[colorlinks = true,
linkcolor = black]{hyperref}
\usepackage{color}
\usepackage[normalem]{ulem}
\usepackage{scrpage2}
\usepackage{graphicx}
\usepackage{tabularx}
\graphicspath{ {../22_Grafiken/01_Logo/}{images/}{../../22_Grafiken/01_Logo/} }
\pagestyle{scrheadings}

\clearscrheadfoot
\ihead{\includegraphics[scale=0.3]{SDDC}}
\ohead{Projekt: SDDC}
\ifoot{Template}
\cfoot{Version: 1.01}
\ofoot{Datum: \today{}}
\setheadsepline{0.5pt}
\setfootsepline{0.5pt}

\usepackage{ucs}
\usepackage[utf8]{inputenc}
\usepackage[T1]{fontenc}


\begin{document}
\def\arraystretch{1.5}
\begin{titlepage}
\begin{center}
\vspace{10em}
\includegraphics[scale=2]{SDDC}
\vspace{10em}
\end{center}
\begin{center}
\huge {Anforderungen Analyse}
\end{center}
\begin{center}
\vspace{10em}
\LARGE {Silvan Adrian} \\
\LARGE {Fabian Binna}
\end{center}

\end{titlepage}

\newpage
\section{Änderungshistorie}
\begin{tabularx}{\linewidth}{l l X l}
\textbf{Datum} & \textbf{Version} & \textbf{Änderung}  & \textbf{Autor} \\
\hline
\textbf{02.10.15} & 1.00 & Erstellung des Dokuments & Gruppe \\
\textbf{02.10.15} & 1.01 & Einführung + Anforderungen Dashboard & Silvan Adrian\\
\end{tabularx}

\newpage
\tableofcontents
\newpage

\section{Einführung}
\subsection{Zweck}
Dieses Dokument beinhaltet die Anforderung zur Analyse.
\subsection{Gültigkeitsbereich}
Dieses Dokument ist während des ganzen Projekts gültig.


\subsection{Referenzen}
-

\section{Anforderungen}
\subsection{API}
\begin{itemize}
\item Die API sollte auf einem möglichst stabilen Stand sein.
\item Es müssen die wichtigsten Provider zur Verfügung stehen (OpenStack, Azure, CloudStack).
\item Die API muss gut dokumentiert sein.
\item Es sollen verschiedene Services angesprochen werden können (Compute, Storage, Network...).
\item Keine grosse Einarbeitung, das heisst die Programmiersprache sollte nicht komplett neu sein.
\end{itemize}

Zudem sind alle Eingenschaften die das Implementieren der Software erleichtern ein Pluspunkt. Von Vorteil währen zusätzliche Funktionen wie z.B SSL oder Pricing. 

\subsection{Dashboard}
Das Dashboard soll eine Möglichkeit für Benutzer bieten, um einzelne Services 
abonnieren zu können.
Dabei soll sowohl IaaS, PaaS oder SaaS abonniert werden können und eine Auswahl 
bieten unter vielen verschiedenen Cloud Anbietern wählen zu können (so generisch wie 
möglich).
Dabei soll der User zwischen einzelnen Angeboten der Anbieter spezifischen 
Services zu wählen bspw.: bei Google Cloud: Cloud DNS, Firewall, Netzwerke etc.
Es kann daher auch sein das nicht alle Anbieter die gleichen Services bieten und 
daher eine Auswahl gegeben werden muss, damit der Benutzer selbst entscheiden 
kann welchen Service er haben will.

\subsubsection{SDDC}
Unser Projekt soll deshalb eine einiges generische Möglichkeit bieten, um 
Service abonnieren zu können und wenn möglich so gut wie alle Cloud Anbieter zu 
unterstützen.
Dies soll möglich werden indem ein Dashboard eine generische API anspricht und 
die API alle Schritte durchführt, die nötig sind für die Erstellung des 
Services.




\end{document}