\documentclass[11pt]{scrartcl}

\title{User Stories Skizzen}
\author{Silvan Adrian \\ Fabian Binna}
\date{\today{}}

\usepackage[ngerman]{babel}
\usepackage[automark]{scrpage2}
\usepackage[colorlinks = true,
linkcolor = black]{hyperref}
\usepackage{color}
\usepackage[normalem]{ulem}
\usepackage{scrpage2}
\usepackage{graphicx}
\usepackage{tabularx}
\graphicspath{ {../22_Grafiken/01_Logo/}{images/}{../../22_Grafiken/01_Logo/} }
\pagestyle{scrheadings}

\clearscrheadfoot
\ihead{\includegraphics[scale=0.3]{SDDC}}
\ohead{Projekt: SDDC}
\ifoot{User Stories Skizzen}
\cfoot{Version: 1.02}
\ofoot{Datum: \today{}}
\setheadsepline{0.5pt}
\setfootsepline{0.5pt}

\usepackage{ucs}
\usepackage[utf8]{inputenc}
\usepackage[T1]{fontenc}


\begin{document}
\def\arraystretch{1.5}
\begin{titlepage}
\begin{center}
\vspace{10em}
\includegraphics[scale=2]{SDDC}
\vspace{10em}
\end{center}
\begin{center}
\huge {User Stories Skizzen}
\end{center}
\begin{center}
\vspace{10em}
\LARGE {Silvan Adrian} \\
\LARGE {Fabian Binna}
\end{center}

\end{titlepage}

\newpage
\section{Änderungshistorie}
\begin{tabularx}{\linewidth}{l l X l}
\textbf{Datum} & \textbf{Version} & \textbf{Änderung}  & \textbf{Autor} \\
\hline
\textbf{26.09.15} & 1.00 & Erstellung des Dokuments & Gruppe \\
\textbf{26.09.15} & 1.01 & Einführung + Rollen & Silvan Adrian \\
\textbf{27.09.15} & 1.02 & Ziele + kurze User Stories Skizzen & Silvan Adrian\\
\end{tabularx}

\newpage
\tableofcontents
\newpage
\section{Einführung}

\subsection{Zweck}

Dieses Dokument beinhaltet die ersten Skizzen der User Stories.

\subsection{Gültigkeitsbereich}

Dieses Dokument ist während des ganzen Projekts gültig.


\subsection{Referenzen}
Bitnami-Analyse.pdf\\
Dashboard-Analyse.pdf\\
UseCase-Skizzen.pdf

\section{Rollen}
\subsection{Public User}
Public User sind alle öffentlichen Besucher des Dashboards.

\subsection{Registered User}
Der registrierte Nutzer ist Anwender des Dashboards und verwendet dieses zur 
Aufgabenerleichterung.
Bei dem Nutzer kann es sich um einen System Administrator, DevOps, Operator oder
Software Entwickler handlen, da beim Dashboard für jeden was dabei ist.

\subsection{Admin}
Der Admin ist für die Instandhaltung des Dashboards zuständig und verwaltet die 
User.

\section{Ziele}
Im Umfang soll die Applikation in etwa folgendes bieten:
\begin{itemize}
  \item Registrierung (Mail Adresse/Passwort)
  \item Login
  \item Administrationoberfläche
  \item Benutzerinfos anpassen
  \item Auswahl aus mehreren Cloud Anbieter
  \item mehrere Cloud Accounts hinzufügen
  \item Abonnieren von Services (Compute/Storage/Network)
  \item Unterteilung der Services in Compute/Storage/Network
  \item Übersicht aller zur Verfügung stehenden Services
  \item Management der Services (erstellen/ändern/löschen)
  \item Links zu Loginpanels von Cloud Anbieter
  \item Übersicht über abonnierte Services
  \item Unterstützung Private Cloud (OpenStack,CloudStack, Docker(Deis -> PaaS))
  \item Anbieter spezifische Services anbieten
  \item generische API
  \item Anstehende Kosten anzeigen
  \item Einfaches hinzufügen eines Cloud Accounts (Wizard bieten)
\end{itemize}


\section{Epic}
\begin{itemize}
  \item Service abonnieren (Compute/Storage/Network)
\end{itemize}

\section{User Stories}

\subsection{Public User}
\begin{itemize}
  \item Als Public User möchte ich mich registrieren können
  \item Als Public User möchte ich mich auf Dashboard verbinden, um einloggen zu 
  können
\end{itemize}
\subsection{Registered User}
\begin{itemize}
  \item Als registered User möchte ich mich einloggen können
  \item Als registered User möchte ich eine Übersicht aller angebotenen Cloud 
  Anbieter sehen
  \item Als registered User möchte ich Zugriff auf meine Accountinfos
  \item Als registered User möchte ich meine Accountinfos anpassen können
  \item Als registered User möchte ich ein Cloud Anbieter auswählen können, um 
  auf die Übersicht der Offerings zu kommen
  \item Als registered User möchte ich eine Übersicht meiner abonnierten 
  Services haben
  \item Als registered User möchte ich Services löschen können
  \item Als registered User möchte ich Compute Instanzen neustarten können
  \item Als registered User möchte ich Compute Instanzen herunterfahren können
  \item Als registered User möchte ich die kosten der Services angezeigt haben
  \item Als registered User möchte ich direkte Verlinkungen zu den Services 
  haben
\end{itemize}

 
\subsection{Admin}
\begin{itemize}
  \item Als Admin möchte ich Zugriff auf eine Administrationsoberfläche
  \item Als Admin möchte ich User erstellen können
  \item Als Admin möchte ich User löschen können
  \item Als Admin möchte ich User ändern können
\end{itemize}


\end{document}