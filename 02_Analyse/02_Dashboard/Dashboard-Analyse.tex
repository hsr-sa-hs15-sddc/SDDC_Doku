\documentclass[11pt]{scrartcl}

\title{Dashboard Analyse}
\author{Silvan Adrian \\ Fabian Binna}
\date{\today{}}

\usepackage[ngerman]{babel}
\usepackage[automark]{scrpage2}
\usepackage[colorlinks = true,
linkcolor = black]{hyperref}
\usepackage{color}
\usepackage[normalem]{ulem}
\usepackage{scrpage2}
\usepackage{graphicx}
\usepackage{tabularx}
\graphicspath{ {../22_Grafiken/01_Logo/}{images/}{../../22_Grafiken/01_Logo/} }
\pagestyle{scrheadings}

\clearscrheadfoot
\ihead{\includegraphics[scale=0.3]{SDDC}}
\ohead{Projekt: SDDC}
\ifoot{Template}
\cfoot{Version: 1.00}
\ofoot{Datum: \today{}}
\setheadsepline{0.5pt}
\setfootsepline{0.5pt}

\usepackage{ucs}
\usepackage[utf8]{inputenc}
\usepackage[T1]{fontenc}


\begin{document}
\def\arraystretch{1.5}
\begin{titlepage}
\begin{center}
\vspace{10em}
\includegraphics[scale=2]{SDDC}
\vspace{10em}
\end{center}
\begin{center}
\huge {Dashboard Analyse}
\end{center}
\begin{center}
\vspace{10em}
\LARGE {Silvan Adrian} \\
\LARGE {Fabian Binna}
\end{center}

\end{titlepage}

\newpage
\section{Änderungshistorie}
\begin{tabularx}{\linewidth}{l l X l}
\textbf{Datum} & \textbf{Version} & \textbf{Änderung}  & \textbf{Autor} \\
\hline
\textbf{17.09.15} & 1.00 & Erstellung des Dokuments & Gruppe \\
\textbf{25.09.15} & 1.01 & Einführung, Security etc. & Silvan Adrian \\
\textbf{26.09.15} & 1.02 & Beschreibung der einzelnen Offerings und Wireframes 
dazu & Silvan Adrian\\

\end{tabularx}

\newpage
\tableofcontents
\newpage

\section{Einführung}
\subsection{Zweck}
Dieses Dokument beinhaltet die Analyse für das eigenes Dashboard.
\subsection{Gültigkeitsbereich}
Dieses Dokument ist während des ganzen Projekts gültig.

\subsection{Referenzen}
Bitnami-Analyse.pdf\\
\href{https://protonmail.ch}{https://protonmail.ch (Security)} 

\section{Dashboard}
Das Dashboard soll dem Nutzer schnell und einfach die Übersicht über seine 
eigenen abonnierten Services bieten (IaaS,PaaS,SaaS).
Dabei soll auf eine Anzahl von Cloud Anbietern zugegriffen werden, sowohl Public 
Cloud(AWS, Google Cloud, Azure, Digitalocean), wie auch Private Cloud (CloudStack.Open 
Stack).
Beim Login wird zwischen Nutzern und Administrator unterschieden, dabei können 
Admin User ändern/löschen/erstellen.



\subsection{Services}
Am Besten auch in etwa gleich gestalten, wie Bitnami und dem User eine Auswahl 
von Services geben zu dem jeweiligen Cloud Anbieter, welchen er schliesslich 
abonnieren kann falls gewünscht.

Füge Bild ein (noch ein paar Wireframes machen, damit sich wenigsten etwas darunter vorstellen kann)

\subsection{Kategorien}
\subsubsection{Compute}
Hier werden nur Compute Offerings angezeigt z.B.: App Engine, Compute Engine, 
Container Engine, EC2 etc., können nach Anbieter varieieren


\subsubsection{Storage}
Nur Storage spezifische Offerings anzeigen z.B.: Cloud Datastore, Cloud SQL, 
Cloud BigTable), die sich je nach Anbieter ändern.



\subsubsection{Network}
Network spezifische Offerings anbieten (Firewall, VPN, Netzwerke, Cloud DNS etc.) 
und dann verändert sich die Auswahl auch Anbieter spezifisch.




\subsection{Accounts/Subscriptions/Projects/...}
Für jeden Anbieter soll dem User eine Übersicht über die 
Account/Subscriptions/Projects gegeben werden, dadurch vereinfacht sich die 
Handhabung von mehreren Accounts und alle sind in einem Dashboard zusammengefügt 
(-> Security beachten).

Nochmals ein Wireframe oder was in der Art zum zeigen!!



\section{Security}
Wie bei Bitnami wäre es wohl sicherer die Zugriffsdaten für die Cloud Anbieter 
abzusichern (bei Bitnami wird dies über ein Vault sichergestellt), ansonsten 
könnte ein Angreifer ganze Instanzen bei verschiedenen Anbietern löschen oder 
sonstige Bösartige Absichten ausüben.

Dieser Vault soll auch durch ein zusätzliches Passwort geschützt sein und 
wird symmetrisch verschlüsselt (Mail Anbieter: Proton Mail macht dies ebenfalls 
so).

\end{document}