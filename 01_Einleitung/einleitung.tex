\part{Einführung}
\chapter{Einleitung}
\begin{center}
  \includegraphics[width=0.6\textwidth]{./22_Grafiken/01_Logo/SDDC}
\end{center}

\section{Aufgabenstellung}

Unter "Software Defined" versteht man die Zentralisierung der Intelligenz in Kontrollern. 
Gerade moderne Data Center werden immer mehr von Software Kontrollern gesteuert, 
damit die Dynamik der Bereitstellung von neuen Services massiv erhöht werden kann. 
So gibt es bereits Kontroller für Storage, Netzwerk und Compute Ressourcen.

Ziel ist es, die Ressourcen Storage, Network und Compute abstrahiert als skalierbare 
Pools der "Service Ebene" zu Verfügung zu stellen. Alle modernen Kontroller können
 über API's angesprochen werden, allerdings unterscheiden sich hier die verschiedenen 
 Hersteller zum Teil stark.
Ziel dieser Arbeit ist die Entwicklung einer generischen Middleware/API, um verschiedene
 Kontroller möglichst einfach in Business Applikationen zu integrieren. Nach der Definition 
 einer systemunabhängigen Schnittstelle sollen die verschiedenen Kontroller danach als 
 Treiber an die API angehängt werden können. Zur Demonstrationszwecken soll eine rudimentäre, 
 ca. 3 Seitige Webpage erstellt werden, welche die erstelle 
API benützt. Dabei soll je mind. ein Storage, Compute und Network Kontroller eingebunden werden.

\section{Vorbemerkungen}
TODO
\section{Zweck}
Diese Arbeit soll die Möglichkeiten aufzeigen, um \ac{SDDC} in möglichst einfacher 
und generischer Art zu betreiben.
TODO----
