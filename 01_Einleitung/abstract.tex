\chapter*{Abstract}\addcontentsline{toc}{chapter}{Abstract}

Unter „Software Defined“ versteht man die Zentralisierung der Intelligenz in Kontrollern. Gerade moderne Data Center werden immer häufiger von Software Kontrollern gesteuert, damit die Dynamik der Bereitstellung von neuen Services massiv erhöht werden kann. Ziel ist, die Ressourcen Storage, Network und Compute abstrahiert als skalierbare Pools der „Service Ebene“ zur Verfügung zu stellen.\\
Eine RESTful API, die den Umgang mit Services, die wiederum Pakete von Ressourcen darstellen, sorgt für einen zentralen Punkt, an den diverse Systeme und Business Applikationen anknüpfen können. Damit die breite Auswahl von Libraries und Produkten in einem Data Center angesprochen werden kann, verwaltet eine generische API die Kontroller und ermöglicht den abstrakten Umgang mit Ressourcen. Die beiden abstrakten Ebenen, RESTful API und generische API, werden mit einem Workflow verbunden. Der Workflow kümmert sich um den zeitlich korrekten Ablauf der Instantiierung. Die Software kann als Webservice in einem Docker Container ausgerollt werden und benötigt danach nur noch eine Konfiguration der generischen API und der Kontroller.