\begin{Protokoll}{Weekly Meeting 5. Oktober}
\moderation{Urs Baumann}
\protokollant{Silvan Adrian, Fabian Binna}
\teilnehmer{Silvan Adrian, Fabian Binna  Urs Baumann}
\fehlendEntschuldigt{Beat Stettler,}
\sitzungsdatum{5.\ Oktober 2015}
\sitzungsbeginn{14:30}
\sitzungsende{15:45}
\sitzungsort{Raum 2.104 }
%\verteiler{Vereinsmitglieder}
%\fehlend[entschuldigt]{abwesend}
%%\fehlendEntschuldigt{}
%%\fehlendUnentschuldigt{}
\protokollKopf

\topic{Anforderungen}%<-- hier Tagesordnungspunkt einfuegen
\subtopic{Cloud Anbieter Accounts}
\begin{itemize}
  \item Einfach mit Test Accounts auskommen (da API eher Firmenintern benötigt wird)
  \item Zugriff auf eine VMware vSphere soll uns noch ermöglicht werden,falls 
  nötig
  \item DevStack für OpenStack VM mit allem nötigen aufgesetzt
\end{itemize}

\subtopic{Storage Anbieter}
\begin{itemize}
  \item Viper Storage
  \item NetApp
  \item etc.
  \item Direkt einbinden wenn möglich über OpenStack oder CloudStack sonst 
  libcloud erweitern
\end{itemize}

\subtopic{Network API}
\begin{itemize}
  \item OpenDaylight
  \item Wichtigster Nenner finden (OpenDaylight) und diesen entweder über 
  OpenStack oder libcloud erweitern
\end{itemize}

\subtopic{VMware}
\begin{itemize}
  \item bessere Unterstützung von VMware vSphare
  \item Ebenfalls über OpenStack Treiber falls möglich sonst libcloud erweitern
\end{itemize}

\subtopic{Templates}
\begin{itemize}
  \item Templates sollen auf Anbieter Seite gespeichert werden (kann je nach 
  Anbieter zu einem Problem werden)
  \item Für Anbieter wo man keine eigenen Templates hinterlegen kann je nachdem 
  auf Scriptfiles/CloudInit setzen
\end{itemize}

\subtopic{libcloud}
\begin{itemize}
  \item Wenn möglich Libcloud erweitern (mit den fehlenden Storage und Network Drivern)
\end{itemize}

\subtopic{CloudStack/OpenStack}
\begin{itemize}
  \item Treiber von OpenStack oder CloudStack verwenden
  \item Merge aus libcloud und OpenStack/CloudStack
\end{itemize}

\topic{Termine und Aufgaben}
\subtopic{Termine}
\termin{2015/10/12}[14:30-15:15]{Projektmeeting im Bau 2.104}

\topic{Aufgaben}
\aufgabe*{Anforderungen verbessern}
\aufgabe*{Prototyp erweitern}
\aufgabe*{OpenStack Heat einarbeiten}
\aufgabe*{OpenStack Treiber suchen und testen ob gemäss Vorgaben möglich}
\aufgabe*{einarbeiten Django + Celery}
\aufgabe*{Libcloud im Detail analysieren}
\end{Protokoll}

