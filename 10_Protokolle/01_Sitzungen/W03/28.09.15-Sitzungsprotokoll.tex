\begin{Protokoll}{Weekly Meeting 28. September}
\moderation{Beat Stettler}
\protokollant{Silvan Adrian, Fabian Binna}
\teilnehmer{Silvan Adrian, Fabian Binna, Beat Stettler, Urs Baumann}
\sitzungsdatum{28.\ September 2015}
\sitzungsbeginn{14:45}
\sitzungsende{15:45}
\sitzungsort{Raum 2.104 }
%\verteiler{Vereinsmitglieder}
%\fehlend[entschuldigt]{abwesend}
%%\fehlendEntschuldigt{}
%%\fehlendUnentschuldigt{}
\protokollKopf

\topic{AnalyseDokumente}%<-- hier Tagesordnungspunkt einfuegen
\subtopic{Use Cases/User Stories}
\begin{itemize}
  \item zu detailiert und viele Funktionen, die nicht nötig sind
  \item Login/Registration etc. noch nicht nötig (erst in Business Applikation)
\end{itemize}

\subtopic{Dashboard/Administrationoberfläche}
\begin{itemize}
  \item Katalog von Services in einem zusammen gepackt (Datenbank/Instanz -> LAMP Stack)
  \item Administrationsoberfläche um Servicekatalog zu bearbeiten/erweitern (auf REST API Ebene)
  \item OpenStack Heat (zur Hinterlegung von Konfigurationen von OpenStack) (jaml)
  \item Rest Schnittstelle feste Namen /Aufrufe (z.B.: Aufruf /LAMP-Stack auf API -> macht alles was benötigt wird)
  \item Authentication erlauben über API Tokens (-> Benutzername + API Token für Cloud Anbieter 
  muss trotzdem in Business Applikation gespeichert werden, oder bessere Lösung finden)
\end{itemize}

\subtopic{Analyse API}
\begin{itemize}
  \item Anforderungen (Dokument erstellen für Anforderungen)
  \item Matrix (für Pro/Kontra -> einfacherer Überblick)
  \item für Gegenleser (Glossar, Anforderungen -> damit nachvollzogen werden wie 
  man auf die Analyse gekommen ist)
  \item Risikoanalyse (zu wenig Stunden)
\end{itemize}

\subtopic{libcloud/Python}
\begin{itemize}
  \item Celery Task Queue
  \item Django REST Framework
  \item Bei Fragen Fässler Christian im INS (20\% angestellt)
  \item Workflow für Service erstellung (Config-File)
\end{itemize}

\topic{Termine und Aufgaben}
\subtopic{Termine}
\termin{2015/10/05}[14:30-15:15]{Projektmeeting im Bau 2.104}

\topic{Aufgaben}
\aufgabe*{Projektplan verbessern}
\aufgabe*{Prototyp für libcloud (über Commandline ansprechbar)}
\aufgabe*{Anforderungen zu Analyse aufschreiben und in ein Dokument packen}
\aufgabe*{Use Cases/User Stories}
\aufgabe*{Requirements}
\aufgabe*{einarbeiten Django + Celery}
\end{Protokoll}
