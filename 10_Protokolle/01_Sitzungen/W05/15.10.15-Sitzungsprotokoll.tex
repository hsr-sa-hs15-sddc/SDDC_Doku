\begin{Protokoll}{Weekly Meeting 15. Oktober}
\moderation{Beat Stettler}
\protokollant{Silvan Adrian, Fabian Binna}
\teilnehmer{Silvan Adrian, Fabian Binna,  Urs Baumann,Beat Stettler}
\sitzungsdatum{15.\ Oktober 2015}
\sitzungsbeginn{08:45}
\sitzungsende{09:30}
\sitzungsort{Raum C2.101 }
%\verteiler{Vereinsmitglieder}
%\fehlend[entschuldigt]{abwesend}
%%\fehlendEntschuldigt{}
%%\fehlendUnentschuldigt{}
\protokollKopf

\topic{Konzept und weiteres Vorgehen}%<-- hier Tagesordnungspunkt einfuegen
\subtopic{Konzept}
\begin{itemize}
  \item Umsetzung in Java
  \item Libary libvirt verwenden und je nachdem noch jclouds für Public Cloud 
  Anbieter
\end{itemize}

\subtopic{Netwerk}
\begin{itemize}
  \item Floodlight
  \item Besser Floodlight oder anderer Controller, da es direkt auf OpenFlow Ebene zu schwierig wird
  \item HP Netzwerk Controller, welcher auf OpenFlow aufbaut ist an der HSR 
  vorhanden
  \item Sonst mit Mininet testen (erlaubt Erstellung von virtuellen Netzwerken) 
\end{itemize}

\subtopic{Grundgerüst}
\begin{itemize}
  \item Alles möglichst einfach halten
  \item Logindaten für Hypervisor über Konfigurationsfile übergeben
\end{itemize}

\subtopic{Konfigurationsfile}
\begin{itemize}
  \item wenn möglich ähnlich aufbauen wie Warm (für OpenStack), wo Datei 
  Workflow vorgibt
\end{itemize}

\subtopic{Einschränkungen}
\begin{itemize}
	\item Keine Authentifizierung
	\item Keine Verteilte Software (nur Client - Server)
\end{itemize}

\topic{Termine und Aufgaben}
\subtopic{Termine}
\termin{2015/10/22}[08:20-09:00]{Projektmeeting im Bau C2.101}

\topic{Aufgaben}
\aufgabe*{Anforderungen verbessern}
\aufgabe*{Projektplan verbessern}
\aufgabe*{libvirt Prototyp erstellen}
\aufgabe*{Design Dokumente erstellen}
\aufgabe*{Planung 1.Sprint}
\aufgabe*{KVM Umgebung aufsetzen}
\end{Protokoll}
