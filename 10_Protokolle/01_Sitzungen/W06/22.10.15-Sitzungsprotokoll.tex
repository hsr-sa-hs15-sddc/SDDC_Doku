\begin{Protokoll}{Weekly Meeting 22. Oktober}
\moderation{Beat Stettler}
\protokollant{Silvan Adrian, Fabian Binna}
\teilnehmer{Silvan Adrian, Fabian Binna,Beat Stettler}
\fehlendEntschuldigt{Urs Baumann}
\sitzungsdatum{22.\ Oktober 2015}
\sitzungsbeginn{08:25}
\sitzungsende{09:55}
\sitzungsort{Raum C2.101 }
%\verteiler{Vereinsmitglieder}
%\fehlend[entschuldigt]{abwesend}
%%\fehlendEntschuldigt{}
%%\fehlendUnentschuldigt{}
\protokollKopf

\topic{Design/Architektur}%<-- hier Tagesordnungspunkt einfuegen
\subtopic{Generic API}
\begin{itemize}
  \item Generic API Facade noch generischer (createResource/deleteResource)
  \item Die Generische API könnte man einzeln verteilen und repräsentiert eine Location.
\end{itemize}

\subtopic{Services verwalten}
\begin{itemize}
  \item Update fehlt momentan noch ganz (gemäss Projektplan oder Sprint nur kündigen und abonnieren)
\end{itemize}

\topic{Abgabe}
\subtopic{Dokument}
\begin{itemize}
  \item Sachlich geschrieben
  \item Ob 1 Dokument oder mehrere spielt keine grosse Rolle
\end{itemize}

\subtopic{Präsentation}
\begin{itemize}
  \item Zum auf oder abrunden (je nach Person und falls auffällt das jemand in einem Bereich weniger Ahnung hat)
\end{itemize}

\topic{Termine und Aufgaben}
\subtopic{Termine}
\termin{2015/10/22}[09:00-09:45]{Projektmeeting im Bau C2.101}

\topic{Aufgaben}
\aufgabe*{Tasks zu User Stories erstellen}
\aufgabe*{Metriken aufsetzen (SonarQube)}
\aufgabe*{Dokumente auf den neusten Stand bringen}
\aufgabe*{Design ausbauen}
\aufgabe*{Grundgerüst aufbauen (Building, Packages, Klassen, Interfaces)}

\end{Protokoll}