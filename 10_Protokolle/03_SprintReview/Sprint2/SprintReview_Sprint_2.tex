\documentclass[11pt]{scrartcl}

\title{Sprint Review}
\author{Silvan Adrian \\ Fabian Binna}
\date{\today{}}

\usepackage[ngerman]{babel}
\usepackage[automark]{scrpage2}
\usepackage{hyperref}
\usepackage{color}
\usepackage[normalem]{ulem}
\usepackage{scrpage2}
\usepackage{graphicx}
\usepackage{tabularx}
\graphicspath{ {./images/} }
\pagestyle{scrheadings}

\clearscrheadfoot

\ohead{Projekt: SDDC}
\ifoot{Sprint Review}
\cfoot{Version: 1.00}
\ofoot{Datum: 23. November 2015}
\setheadsepline{0.5pt}
\setfootsepline{0.5pt}

\usepackage{ucs}
\usepackage[utf8]{inputenc}
\usepackage[T1]{fontenc}


\begin{document}
\def\arraystretch{1.5}
\begin{titlepage}
\begin{center}
\vspace{10em}

\vspace{10em}
\end{center}
\begin{center}
\huge {Sprint Review}\\

Durchgeführt am 23. November 2015\\
Sprint 2
\end{center}

\end{titlepage}

\newpage
\tableofcontents
\newpage

\section{Ziele}
Restful Interface für die API, um Services über HTTP 
abonnieren zu können + User-Dashboard (sehr spartanisch)  um Services 
abonnieren/kündigen zu können.

\section{User Stories}

 \subsubsection{Customer geht in die Offerings Übersicht}
\begin{tabularx}{\linewidth}{l X}
  \textbf{Priorität} & Hoch\\
  \hline
  \textbf{Story Points} & 6\\
  \hline
  \textbf{Story}& Als Customer möchte ich die Offerings in einer Übersicht ansehen können\\
  \hline
    \textbf{Akzeptanzkriterien} & \\
    \hline
  \textbf{A1} & Der Customer kann im Dashboard in die Offerings Übersicht wechseln.\\
  \hline
   \end{tabularx}
   
 \subsubsection{Customer geht in die Service Übersicht}
\begin{tabularx}{\linewidth}{l X}
  \textbf{Priorität} & Hoch\\
  \hline
  \textbf{Story Points} & 6\\
  \hline
  \textbf{Story}& Als Customer möchte ich meine abonnierten Services in einer Übersicht angezeigt bekommen\\
  \hline
    \textbf{Akzeptanzkriterien} & \\
    \hline
  \textbf{A1} & Der Customer kann seine abonnierten Services in einer Übersicht anzeigen\\
  \hline
   \end{tabularx}
   

 \subsubsection{Admin erstellt Service}
 \begin{tabularx}{\linewidth}{l X}
  \textbf{Priorität} & Hoch\\
  \hline
  \textbf{Story Points} & 6\\
  \hline
  \textbf{Story}& Als Admin möchte ich Services erstellen können\\
  \hline
    \textbf{Akzeptanzkriterien} & \\
    \hline
      \textbf{A1} & Der Service kann nur erstellt werden, falls Admin eingeloggt ist.\\
  \hline
  \textbf{A2} & Als Admin krieg ich die Übersicht der verfügbaren Services\\
  \hline
    \textbf{A3} & Dem Service können Servicemodule hinzugefügt werden\\
  \hline
  \textbf{A4} & Service kann erstellt werden\\
  \hline
    \textbf{A5} & Der Service ist erstellt\\
  \hline
 \end{tabularx}
 
 \subsubsection{Admin ändert Service}
 \begin{tabularx}{\linewidth}{l X}
  \textbf{Priorität} & Hoch\\
  \hline
  \textbf{Story Points} & 6\\
  \hline
  \textbf{Story}& Als Admin möchte ich Services ändern können\\
  \hline
    \textbf{Akzeptanzkriterien} & \\
    \hline
      \textbf{A1} & Der Service kann nur geändert werden, falls Admin eingeloggt ist.\\
  \hline
  \textbf{A2} & Als Admin krieg ich die Übersicht der verfügbaren Services\\
  \hline
    \textbf{A3} & Dem Service können Servicemodule hinzugefügt werden\\
  \hline
  \textbf{A4} & Service kann geändert werden\\
  \hline
    \textbf{A5} & Der Service ist geändert\\
  \hline
  \textbf{A6} & Der Service wird als neue Version gespeichert\\
  \hline
 \end{tabularx}

 \subsubsection{Admin löscht Service}
 
  \begin{tabularx}{\linewidth}{l X}
  \textbf{Priorität} & Hoch\\
  \hline
  \textbf{Story Points} & 6\\
  \hline
  \textbf{Story}& Als Admin möchte ich Services löschen können\\
  \hline
    \textbf{Akzeptanzkriterien} & \\
    \hline
      \textbf{A1} & Der Service kann nur gelöscht werden, falls Admin eingeloggt ist.\\
  \hline
  \textbf{A2} & Als Admin krieg ich die Übersicht der verfügbaren Services\\
  \hline
  \textbf{A4} & Service kann gelöscht werden, falls niemand mehr den Service abonniert hat\\
  \hline
    \textbf{A5} & Der Service ist gelöscht\\
  \hline
 \end{tabularx}


\newpage

\section{Systemtestprotokoll}

siehe Systemtestprotokoll für 2. Sprint

\section{Kommentare}

Siehe Code Review für 2.Sprint

\end{document}