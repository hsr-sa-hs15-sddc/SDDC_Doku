\documentclass[11pt]{scrartcl}

\title{Sprint Review}
\author{Silvan Adrian \\ Fabian Binna}
\date{\today{}}

\usepackage[ngerman]{babel}
\usepackage[automark]{scrpage2}
\usepackage{hyperref}
\usepackage{color}
\usepackage[normalem]{ulem}
\usepackage{scrpage2}
\usepackage{graphicx}
\usepackage{tabularx}
\graphicspath{ {./images/} }
\pagestyle{scrheadings}

\clearscrheadfoot

\ohead{Projekt: SDDC}
\ifoot{Sprint Review}
\cfoot{Version: 1.00}
\ofoot{Datum: 07. Dezember 2015}
\setheadsepline{0.5pt}
\setfootsepline{0.5pt}

\usepackage{ucs}
\usepackage[utf8]{inputenc}
\usepackage[T1]{fontenc}


\begin{document}
\def\arraystretch{1.5}
\begin{titlepage}
\begin{center}
\vspace{10em}

\vspace{10em}
\end{center}
\begin{center}
\huge {Sprint Review}\\

Durchgeführt am 07. Dezember 2015\\
Sprint 3
\end{center}

\end{titlepage}

\newpage
\tableofcontents
\newpage

\section{Ziele}
Admin Dashboard, um Services/Servicemodule 
verwalten und ihnen ein Konfigurationsdatei hinterlegen zu können +
 letzter Release der Software

\section{User Stories}

 \subsubsection{Customer will Informationen über abonnierte Services sehen}
\begin{tabularx}{\linewidth}{l X}
  \textbf{Priorität} & Hoch\\
  \hline
  \textbf{Story Points} & 6\\
  \hline
  \textbf{Story}& Als Customer möchte ich Informationen über abonnierte Service einsehen können.\\
  \hline
    \textbf{Akzeptanzkriterien} & \\
    \hline
  \textbf{A1} & Der Customer kann abonnierten Service auswählen und kriegt Infos zu den Servicemodulen\\
  \hline
 \end{tabularx}
 


 \subsubsection{Admin greift auf Dashboard zu}
\begin{tabularx}{\linewidth}{l X}
  \textbf{Priorität} & Hoch\\
  \hline
  \textbf{Story Points} & 1\\
  \hline
  \textbf{Story}& Als Admin möchte ich auf das Admin-Dashboard zugreifen können\\
  \hline
    \textbf{Akzeptanzkriterien} & \\
    \hline
  \textbf{A1} & Der Admin kann den Url des Customer-Dashboard aufrufen und 
  kriegt ein Login angezeigt.\\
  \hline
 \end{tabularx}
 
 \subsubsection{Admin geht in die Service Übersicht}
 
    \begin{tabularx}{\linewidth}{l X}
  \textbf{Priorität} & Hoch\\
  \hline
  \textbf{Story Points} & 2\\
  \hline
  \textbf{Story}& Als Admin möchte ich einen Überblick über die vorhanden Services\\
  \hline
    \textbf{Akzeptanzkriterien} & \\
    \hline
      \textbf{A1} & Der Admin kann die Service Übersicht öffnen\\
  \hline
 \end{tabularx}


 \subsubsection{Admin Konfigurationsdatei im Servicemodul hinterlegen}
     \begin{tabularx}{\linewidth}{l X}
  \textbf{Priorität} & Hoch\\
  \hline
  \textbf{Story Points} & 2\\
  \hline
  \textbf{Story}& Als Admin möchte ich dem Servicemodul eine Konfigurationsdatei hinterlegen\\
  \hline
    \textbf{Akzeptanzkriterien} & \\
    \hline
      \textbf{A1} & Der Admin kann dem Servicemodul eine Konfigurationsdatei hinterlegen\\
  \hline
 \end{tabularx}
 

 \subsubsection{Admin erstellt Servicemodul}
\begin{tabularx}{\linewidth}{l X}
  \textbf{Priorität} & Hoch\\
  \hline
  \textbf{Story Points} & 4\\
  \hline
  \textbf{Story}& Als Admin möchte ich Servicemodule erstellen können\\
 \hline
    \textbf{Akzeptanzkriterien} & \\
    \hline
  \textbf{A2} & Das Servicemodul wird erstellt und wird in der Servicemodule Übersicht angezeigt\\
  \hline
\end{tabularx}
 
   \subsubsection{Admin ändert Servicemodul}
\begin{tabularx}{\linewidth}{l X}
  \textbf{Priorität} & Hoch\\
  \hline
  \textbf{Story Points} & 4\\
  \hline
  \textbf{Story}& Als Admin möchte ich Servicemodule ändern können\\
  \hline
   \textbf{Akzeptanzkriterien} & \\
    \hline
  \textbf{A1} & Als Admin krieg ich die Übersicht der verfügbaren Servicemodule\\
  \hline
  \textbf{A2} & Als Admin kann ich das Servicemodule anpassen und speichern\\
  \hline
   \textbf{A3} & Änderungen werden gespeichert\\
 \hline
 \end{tabularx}

 
 \subsubsection{Admin löscht Servicemodul}

 \begin{tabularx}{\linewidth}{l X}
 \textbf{Priorität} & Hoch\\
 \hline
  \textbf{Story Points} & 4\\
  \hline
  \textbf{Story}& Als Admin möchte ich Servicemodule löschen können\\
  \hline
   \textbf{Akzeptanzkriterien} & \\
   \hline
   \textbf{A1} & Als Admin krieg ich die Übersicht der verfügbaren Servicemodule\\
   \hline
   \textbf{A2} & Servicemodule ist gelöscht\\
     \hline
\end{tabularx}

 \subsubsection{Admin geht in die Servicemodul Übersicht}

 \begin{tabularx}{\linewidth}{l X}
\textbf{Priorität} & Hoch\\
 \hline
 \textbf{Story Points} & 2\\
 \hline
 \textbf{Story}& Als Admin möchte ich einen Überblick über die vorhanden Servicemodule\\
 \hline
 \textbf{Akzeptanzkriterien} & \\
 \hline
  \textbf{A1} & Der Admin kann die Servicemodule Übersicht öffnen\\
 \hline
\textbf{A2} & Die Servicemodule Übersicht wird angezeigt.\\
  \hline
 \end{tabularx}

\newpage

\section{Testprotokoll}

siehe Testprotokoll für 3. Sprint

\section{Kommentare}

Siehe Code Review für 3.Sprint

\end{document}