
\chapter{Code Review Sprint 3}
\section{Arbeitspakete}

\begin{center}

\begin{tabularx}{\linewidth}{l l l l}
\textbf{ID} & \textbf{Subject} & \textbf{Assignee} & \textbf{Bewertung}\\
\hline
\textbf{153} & Admin erstellt ServiceModule & Team & OK\\
\textbf{154} & Admin löscht ServiceModule & Team & OK\\
\textbf{155} & Admin ändert ServiceModule & Team & OK\\
\textbf{157} & Admin geht in die ServiceModule Übersicht & Team & OK\\
\textbf{158} & Admin geht in die Service Übersicht & Team & OK\\
\textbf{159} & Admin will Konfigurationsdatei im ServiceModule hinterlegen & Team & OK\\
\textbf{177} & Customer will Informationen über abonnierten Service sehen & Team & OK\\
\textbf{187} & Neue GenericAPI & Team & OK\\
\end{tabularx}

\end{center}
\newpage

\section{Kommentare}

\subsection{187, Neue GenericAPI}
Das Dependency Injection File (Config.xml) muss nach dem Deployment noch veränderbar sein.

\subsection{Allgemein}
\subsubsection{Cleanup}
Warnings entfernen und Code aufräumen.

\subsubsection{Logging}
An wichtigen Stellen muss noch geloggt werden.

\subsubsection{Javadoc}
Die GenericAPI benötigt noch Javadoc Einträge um die Implementation neuer Controller zu vereinfachen.
