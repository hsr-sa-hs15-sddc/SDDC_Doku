\documentclass[11pt]{scrartcl}

\title{Anforderungsspezifikation}
\author{Silvan Adrian \\ Fabian Binna}
\date{\today{}}

\usepackage[ngerman]{babel}
\usepackage[automark]{scrpage2}
\usepackage[colorlinks = true,
linkcolor = black]{hyperref}
\usepackage{color}
\usepackage[normalem]{ulem}
\usepackage{scrpage2}
\usepackage{graphicx}
\usepackage{tabularx}
\usepackage{longtable, tabu}
\graphicspath{ {../22_Grafiken/01_Logo/}{images/}{../../22_Grafiken/01_Logo/} }
\pagestyle{scrheadings}

\clearscrheadfoot
\ihead{\includegraphics[scale=0.3]{SDDC}}
\ohead{Projekt: SDDC}
\ifoot{Anforderungsspezifikation}
\cfoot{Version: 1.10}
\ofoot{Datum: \today{}}
\setheadsepline{0.5pt}
\setfootsepline{0.5pt}

\usepackage{ucs}
\usepackage[utf8]{inputenc}
\usepackage[T1]{fontenc}


\begin{document}
\def\arraystretch{1.5}
\begin{titlepage}
\begin{center}
\vspace{10em}
\includegraphics[scale=2]{SDDC}
\vspace{10em}
\end{center}
\begin{center}
\huge {Anforderungsspezifikation}
\end{center}
\begin{center}
\vspace{10em}
\LARGE {Silvan Adrian} \\
\LARGE {Fabian Binna}
\end{center}

\end{titlepage}

\newpage
\section{Änderungshistorie}
\begin{tabularx}{\linewidth}{l l X l}
\textbf{Datum} & \textbf{Version} & \textbf{Änderung}  & \textbf{Autor} \\
\hline
\textbf{02.10.15} & 1.00 & Erstellung des Dokuments & Gruppe \\
\textbf{02.10.15} & 1.01 & Nicht funktionale Anforderungen & Silvan Adrian\\
\textbf{02.10.15} & 1.02 & Use Cases Aktoren + User Stories Aktoren & Silvan 
Adrian\\
\textbf{03.10.15} & 1.03 & Anforderungen API & Fabian Binna\\
\textbf{03.10.15} & 1.04 & Anforderungen Dashboards + Mockups eingefügt & Silvan 
Adrian\\
\textbf{03.10.15} & 1.05 & Use Cases fully dressed & Silvan Adrian\\
\textbf{04.10.15} & 1.06 & User Stories + NFA's verbessert & Silvan Adrian\\
\textbf{17.10.15} & 1.07 & Verbesserungen & Silvan Adrian\\
\textbf{23.10.15} & 1.08 & User Stories erweitert & Silvan Adrian\\
\textbf{31.10.15} & 1.08 & Verbesserungen & Silvan Adrian\\
\textbf{06.11.15} & 1.09 & Gemäss neusten Erkenntnissen verbessert & Silvan Adrian\\
\textbf{27.11.15} & 1.10 & Verbesserungen & Silvan Adrian\\
\end{tabularx}

\newpage
\tableofcontents
\newpage

\section{Einführung}
\subsection{Zweck}
Dieses Dokument beinhaltet die Anforderung zur Analyse.
\subsection{Gültigkeitsbereich}
Dieses Dokument ist während des ganzen Projekts gültig.


\subsection{Referenzen}
-

\section{Anforderungen}


\subsection{API}
\begin{itemize}
\item Die API sollte auf einem möglichst stabilen Stand sein.
\item Es müssen die wichtigsten Provider zur Verfügung stehen.
\item Die API muss gut dokumentiert sein.
\item Es sollen verschiedene Services angesprochen werden können (Compute, Storage, Network...).
\item Keine grosse Einarbeitung, das heisst die Programmiersprache sollte nicht komplett neu sein.
\end{itemize}

Zudem sind alle Eingenschaften die das Implementieren der Software erleichtern ein 
Pluspunkt. Von Vorteil währen zusätzliche Funktionen wie z.B SSL oder Pricing. 

\subsection{User-Dashboard}
Das User-Dashboard soll eine Möglichkeit für Benutzer bieten, um einzelne Services 
abonnieren zu können.
Dabei soll sowohl IaaS, PaaS oder SaaS abonniert werden können und eine Auswahl 
bieten unter vielen verschiedenen Cloud Anbietern wählen zu können (so generisch wie 
möglich).
Dabei soll der User zwischen einzelnen Angeboten der Anbieter spezifischen 
Services zu wählen bspw.: bei Google Cloud: Cloud DNS, Firewall, Netzwerke etc.
Es kann daher auch sein das nicht alle Anbieter die gleichen Services bieten und 
daher eine Auswahl gegeben werden muss, damit der Benutzer selbst entscheiden 
kann welchen Service er haben will.

\subsubsection{SDDC}
Unser Projekt soll deshalb eine einiges generische Möglichkeit bieten, um 
Service abonnieren zu können und wenn möglich so gut wie alle Cloud Anbieter zu 
unterstützen.
Dies soll möglich werden indem ein Dashboard eine generische API anspricht und 
die API alle Schritte durchführt, die nötig sind für die Erstellung des 
Services.

\subsection{Admin-Dashboard}
Dem Admin soll eine Möglichkeit geboten werden um die Software administrieren zu 
können, z.B.: Benutzerverwaltung oder etwas in der Art.


\section{Use Cases}
\subsection{Use Case Diagramm}
\includegraphics[width=0.84\textwidth]{UseCase-Diagramm}
\subsection{Aktoren \& Stakeholders}
\subsubsection{Customer}
Als Customer möchte ich meine abonnierten Services verwalten.
\\
\begin{tabularx}{\linewidth}{l l X }
  \textbf{Aktor} & \textbf{Typ} & \textbf{Ziele}\\
  \hline
  Customer & Primary & 
  \begin{minipage}{5in}
  \vskip 4pt
  \begin{itemize}
    \item Service abonnieren
    \item Service kündigen
  \end{itemize}
  \vskip 4pt
 \end{minipage}\\
 \hline
\end{tabularx}


\subsubsection{Admin}
Als Admin möchte ich Services und Servicemodule verwalten können.
\\
\begin{tabularx}{\linewidth}{l l X }
  \textbf{Aktor} & \textbf{Typ} & \textbf{Ziele}\\
  \hline
  Admin & Primary & 
  \begin{minipage}{5in}
  \vskip 4pt
  \begin{itemize}
    \item Service erstellen
    \item Service anpassen
    \item Service löschen
    \item Servicemodul erstellen
    \item Servicemodul anpassen
    \item Servicemodul löschen
  \end{itemize}
  \vskip 4pt
 \end{minipage}\\
 \hline
\end{tabularx}

\newpage
\subsection{Beschreibungen fully dressed}

\subsubsection{UC01: Service abonnieren}
\belowtabulinesep = 1mm
\begin{longtabu} to \textwidth {X[1,l] X[2,l]}
	\bfseries Primäraktor & Customer  \\\hline 
	\bfseries Steakholders und Interessen & Customer: Möchte einen Service abonnieren  \\\hline 
	\bfseries Vorbedingungen & Das Customer-Dashboard wurde geöffnet. \\\hline 
	\bfseries Nachbedingungen & Die Service Infos wurden gespeichert und der Workflow wurde angestossen  
	\\\hline 
	\bfseries Standartablauf & 
		\begin{enumerate}
			\item Wiederholen bis kein Service mehr abonniert werden soll
			\begin{enumerate}
			    \item Der Customer wechselt in die \textbf{Offerings Übersicht}
			    \item Der Customer wählt einen der vorhanden \textbf{Services} aus
			    \item Der Customer drückt den Button \textbf{subscribe Service}
			    \item Der Customer wird in die \textbf{Services Übersicht} weitergeleitet
			\end{enumerate}
		\item Der Customer schliesst das Customer-Dashboard
		\end{enumerate}
      \\\hline
      	\bfseries Alternativer Ablauf & 
		\begin{enumerate}
		  
		  
		  \item 
                 \begin{enumerate}
		    \item Der Customer entscheidet sich um
		    \begin{enumerate}
		      \item Schliesst das Fenster/Tab
		    \end{enumerate}
		    \item  Der Customer entscheidet sich um
		     \begin{enumerate}
		      \item Schliesst das Fenster/Tab
		      \item geht zurück in die \textbf{Offerings Übersicht}
		    \end{enumerate}
		    
		       \item  Der Customer entscheidet sich um
		     \begin{enumerate}
		      \item Schliesst das Fenster/Tab
		      \item geht zurück in die \textbf{Offerings Übersicht}
		    \end{enumerate}
		    
		    
		     \item  Der Customer entscheidet sich um
		     \begin{enumerate}
		      \item Schliesst das Fenster/Tab
		      \item geht zurück in die \textbf{Offerings Übersicht}
		    \end{enumerate}
		    
		  \end{enumerate}
			
		\end{enumerate}
	 \\\hline
	\bfseries Spezielle Anforderungen & siehe nichtfunktionale Anforderungen  \\\hline 
	\bfseries Technologie- und Datenvarianten & Keine  \\\hline 
	\bfseries Auftrittshäufigkeit & mehrmals pro Woche  \\\hline 
	\bfseries Offene Fragen & Keine  \\\hline  
\end{longtabu}


\newpage

\subsubsection{UC02: Service kündigen}
\begin{longtabu} to \textwidth {X[1,l] X[2,l]}
	\bfseries Primäraktor & Customer  \\\hline 
	\bfseries Steakholders und Interessen & Customer: Möchte einen Service kündigen  \\\hline 
	\bfseries Vorbedingungen & Das Customer-Dashboard wurde geöffnet.  \\\hline 
	\bfseries Nachbedingungen & Die Service Infos wurden gelöscht und der Workflow wurde angestossen   \\\hline 
	\bfseries Standartablauf & 
		\begin{enumerate}
			\item Wiederholen bis kein Service mehr gekündigt werden soll
			\begin{enumerate}
			    \item Der Customer wechselt in die \textbf{Services Übersicht}
			    \item Der Customer wählt einen der vorhanden \textbf{Services} aus
			    \item Der Customer drückt auf den link \textbf{terminate}
			    \item Der Customer wird in die \textbf{Services Übersicht} weitergeleitet
			\end{enumerate}
		\item Der Customer schliesst das Customer-Dashboard
		\end{enumerate}
      \\\hline
      	\bfseries Alternativer Ablauf & 
		\begin{enumerate}
		  
		  \item 
                 \begin{enumerate}
		    \item Der Customer entscheidet sich um
		    \begin{enumerate}
		      \item Schliesst das Fenster/Tab
		    \end{enumerate}
		    \item  Der Customer entscheidet sich um
		     \begin{enumerate}
		      \item Schliesst das Fenster/Tab
		      \item wählt einen anderen Service
		    \end{enumerate}
		    
		       \item  Der Customer entscheidet sich um
		     \begin{enumerate}
		      \item Schliesst das Fenster/Tab
		      \item wählt einen anderen Service
		    \end{enumerate}
		    
		  \end{enumerate}
			
		\end{enumerate}
	 \\\hline

      \\\hline
	\bfseries Spezielle Anforderungen & siehe nichtfunktionale Anforderungen  \\\hline 
	\bfseries Technologie- und Datenvarianten & Keine  \\\hline 
	\bfseries Auftrittshäufigkeit & mehrmals pro Woche  \\\hline 
	\bfseries Offene Fragen & Keine  \\\hline  
\end{longtabu}
%Service kündigen end


\newpage

\subsubsection{UC04: Service verwalten}

\begin{longtabu} to \textwidth {X[1,l] X[2,l]}
	\bfseries Primäraktor & Admin  \\\hline 
	\bfseries Steakholders und Interessen & Admin: Möchte einen Service verwalten  \\\hline 
	\bfseries Vorbedingungen & Das Admin-Dashboard wurde geöffnet
	\\\hline 
	\bfseries Nachbedingungen & Das Admin-Dashboard wurde geschlossen und Änderungen 
	wurden gespeichert  \\\hline 
	\bfseries Standartablauf & 
		\begin{enumerate}
			\item Der Admin öffnet das Admin-Dashboard
			\item Wiederholen bis kein neuer Service hinzugefügt werden muss
			\begin{enumerate}
			  \item Der Admin wechselt in die \textbf{Services Übersicht}
			  \item Der Admin drückt auf den button \textbf{create new Service}
			  \item Der Admin füllt die benötigten Daten ein \textbf{(Name, welche Servicemodule)}
			  \item der Admin bestätigt mit Klick auf Button \textbf{Save}
			\end{enumerate}
		\end{enumerate}
      \\\hline
      \bfseries Alternativer Ablauf & 
      \begin{enumerate}
        \setcounter{enumi}{1}
        \item 
        \begin{enumerate}
          \item Wiederholen bis kein Service mehr geändert werden muss
            \begin{enumerate}
              \item Service auswählen und auf Link \textbf{edit} klicken
              \item Daten ändern
              \item Durch Klick auf Button \textbf{Save} bestätigen
            \end{enumerate}
            \item Wiederholen bis kein Service mehr gelöscht werden muss
            \begin{enumerate}
              \item Service auswählen und Link \textbf{delete} auswählen
            \end{enumerate}
        \end{enumerate}
      \end{enumerate}
      \\\hline
	\bfseries Spezielle Anforderungen & siehe nichtfunktionale Anforderungen  \\\hline 
	\bfseries Technologie- und Datenvarianten & Keine  \\\hline 
	\bfseries Auftrittshäufigkeit & mehrmals pro Monat  \\\hline 
	\bfseries Offene Fragen & Keine  \\\hline  
\end{longtabu}


\newpage

\newpage
\section{Epics}
\subsection{Customer}
\begin{itemize}
  \item Service abonnieren
  \item Service kündigen
\end{itemize}
\subsection{Admin}
\begin{itemize}
  \item Service verwalten
  \item Servicemodul verwalten
\end{itemize}
\section{User Stories}
\subsection{Rollen}
\subsubsection{Customer}
Als Customer benutze ich das Dashboard, um für mich einen Service zu abonnieren oder zu 
kündigen, ebenfalls verwalte ich meine Cloud Login Daten
\subsubsection{Admin}
Als Admin erstelle ich neue Services und Servicemodule und erweitere diese um 
neue Funktionen/Verbesserungen.
\subsection{Customer}

 \subsubsection{Customer abonniert Service}
\begin{tabularx}{\linewidth}{l X}
  \textbf{Priorität} & Hoch\\
  \hline
  \textbf{Story Points} & 4\\
  \hline
  \textbf{Story}& Als Customer möchte ich einen Service abonnieren können\\
  \hline
    \textbf{Akzeptanzkriterien} & \\
    \hline
  \textbf{A1} & Der Customer kann einen Service abonnieren\\
  \hline
    \textbf{A3} & Storage,Compute und Network wurden, wie im Service beschrieben erstellt\\
  \hline  
     \end{tabularx}

 

 \subsubsection{Customer kündigt Service}
 \begin{tabularx}{\linewidth}{l X}
     \textbf{Priorität} & Hoch\\
  \hline
  \textbf{Story Points} & 4\\
  \hline
  \textbf{Story}& Als Customer möchte ich einen Service kündigen können\\
  \hline
    \textbf{Akzeptanzkriterien} & \\
    \hline
  \textbf{A1} & Der Customer kann einen Service kündigen\\
  \hline
    \textbf{A3} & Storage,Compute und Network werden, wie im Service beschrieben gelöscht\\
  \hline
   \end{tabularx}

\subsubsection{Customer will abonnierte Services sehen}
\begin{tabularx}{\linewidth}{l X}
  \textbf{Priorität} & Hoch\\
  \hline
  \textbf{Story Points} & 2\\
  \hline
  \textbf{Story}& Als Customer möchte ich sehen welche Services ich abonniert habe.\\
  \hline
    \textbf{Akzeptanzkriterien} & \\
    \hline
  \textbf{A1} & Der Customer kriegt eine Liste mit seinen abonnierten Services zurück\\
  \hline
   \end{tabularx}
   
\subsubsection{Customer will verfügbare Services angezeigt bekommen}
\begin{tabularx}{\linewidth}{l X}
  \textbf{Priorität} & Hoch\\
  \hline
  \textbf{Story Points} & 2\\
  \hline
  \textbf{Story}& Als Customer möchte ich sehen welche Services ich abonnieren kann.\\
  \hline
    \textbf{Akzeptanzkriterien} & \\
    \hline
  \textbf{A1} & Der Customer kriegt eine Liste mit seinen zur Verfügung stehenden Services zurück\\
  \hline
   \end{tabularx}
   

 \subsubsection{Customer geht in die Offerings Übersicht}
\begin{tabularx}{\linewidth}{l X}
  \textbf{Priorität} & Hoch\\
  \hline
  \textbf{Story Points} & 6\\
  \hline
  \textbf{Story}& Als Customer möchte ich die Offerings in einer Übersicht ansehen können\\
  \hline
    \textbf{Akzeptanzkriterien} & \\
    \hline
  \textbf{A1} & Der Customer kann im Dashboard in die Offerings Übersicht wechseln.\\
  \hline
   \end{tabularx}
   
 \subsubsection{Customer geht in die Service Übersicht}
\begin{tabularx}{\linewidth}{l X}
  \textbf{Priorität} & Hoch\\
  \hline
  \textbf{Story Points} & 6\\
  \hline
  \textbf{Story}& Als Customer möchte ich meine abonnierten Services in einer Übersicht angezeigt bekommen\\
  \hline
    \textbf{Akzeptanzkriterien} & \\
    \hline
  \textbf{A1} & Der Customer kann seine abonnierten Services in einer Übersicht anzeigen\\
  \hline
   \end{tabularx}
   

  \subsubsection{Customer will Informationen über abonnierte Services sehen}
\begin{tabularx}{\linewidth}{l X}
  \textbf{Priorität} & Hoch\\
  \hline
  \textbf{Story Points} & 6\\
  \hline
  \textbf{Story}& Als Customer möchte ich Informationen über abonnierte Service einsehen können.\\
  \hline
    \textbf{Akzeptanzkriterien} & \\
    \hline
  \textbf{A1} & Der Customer kann abonnierten Service auswählen und kriegt Infos zu den Servicemodulen\\
  \hline
 \end{tabularx}
 

  
 \subsection{Admin}
 
  \subsubsection{Admin erstellt Service}
 \begin{tabularx}{\linewidth}{l X}
  \textbf{Priorität} & Hoch\\
  \hline
  \textbf{Story Points} & 6\\
  \hline
  \textbf{Story}& Als Admin möchte ich Services erstellen können\\
  \hline
    \textbf{Akzeptanzkriterien} & \\
    \hline
      \textbf{A1} & Der Service kann nur erstellt werden, falls Admin eingeloggt ist.\\
  \hline
  \textbf{A2} & Als Admin krieg ich die Übersicht der verfügbaren Services\\
  \hline
    \textbf{A3} & Dem Service können Servicemodule hinzugefügt werden\\
  \hline
  \textbf{A4} & Service kann erstellt werden\\
  \hline
    \textbf{A5} & Der Service ist erstellt\\
  \hline
 \end{tabularx}
 
 \subsubsection{Admin ändert Service}
 \begin{tabularx}{\linewidth}{l X}
  \textbf{Priorität} & Hoch\\
  \hline
  \textbf{Story Points} & 6\\
  \hline
  \textbf{Story}& Als Admin möchte ich Services ändern können\\
  \hline
    \textbf{Akzeptanzkriterien} & \\
    \hline
      \textbf{A1} & Der Service kann nur geändert werden, falls Admin eingeloggt ist.\\
  \hline
  \textbf{A2} & Als Admin krieg ich die Übersicht der verfügbaren Services\\
  \hline
    \textbf{A3} & Dem Service können Servicemodule hinzugefügt werden\\
  \hline
  \textbf{A4} & Service kann geändert werden\\
  \hline
    \textbf{A5} & Der Service ist geändert\\
  \hline
  \textbf{A6} & Der Service wird als neue Version gespeichert\\
  \hline
 \end{tabularx}

 \subsubsection{Admin löscht Service}
 
  \begin{tabularx}{\linewidth}{l X}
  \textbf{Priorität} & Hoch\\
  \hline
  \textbf{Story Points} & 6\\
  \hline
  \textbf{Story}& Als Admin möchte ich Services löschen können\\
  \hline
    \textbf{Akzeptanzkriterien} & \\
    \hline
      \textbf{A1} & Der Service kann nur gelöscht werden, falls Admin eingeloggt ist.\\
  \hline
  \textbf{A2} & Als Admin krieg ich die Übersicht der verfügbaren Services\\
  \hline
  \textbf{A4} & Service kann gelöscht werden, falls niemand mehr den Service abonniert hat\\
  \hline
    \textbf{A5} & Der Service ist gelöscht\\
  \hline
 \end{tabularx}

 
  \subsubsection{Admin greift auf Dashboard zu}
\begin{tabularx}{\linewidth}{l X}
  \textbf{Priorität} & Hoch\\
  \hline
  \textbf{Story Points} & 1\\
  \hline
  \textbf{Story}& Als Admin möchte ich auf das Admin-Dashboard zugreifen können\\
  \hline
    \textbf{Akzeptanzkriterien} & \\
    \hline
  \textbf{A1} & Der Admin kann den Url des Customer-Dashboard aufrufen und 
  kriegt das Dashboard angezeigt\\
  \hline
 \end{tabularx}
 
 \subsubsection{Admin geht in die Service Übersicht}
 
    \begin{tabularx}{\linewidth}{l X}
  \textbf{Priorität} & Hoch\\
  \hline
  \textbf{Story Points} & 2\\
  \hline
  \textbf{Story}& Als Admin möchte ich einen Überblick über die vorhanden Services\\
  \hline
    \textbf{Akzeptanzkriterien} & \\
    \hline
      \textbf{A1} & Der Admin kann die Service Übersicht öffnen\\
  \hline
 \end{tabularx}


 \subsubsection{Admin Konfigurationsdatei im Servicemodul hinterlegen}
     \begin{tabularx}{\linewidth}{l X}
  \textbf{Priorität} & Hoch\\
  \hline
  \textbf{Story Points} & 2\\
  \hline
  \textbf{Story}& Als Admin möchte ich dem Servicemodul eine Konfigurationsdatei hinterlegen\\
  \hline
    \textbf{Akzeptanzkriterien} & \\
    \hline
      \textbf{A1} & Der Admin kann dem Servicemodul eine Konfigurationsdatei hinterlegen\\
  \hline
 \end{tabularx}
 

 \subsubsection{Admin erstellt Servicemodul}
\begin{tabularx}{\linewidth}{l X}
  \textbf{Priorität} & Hoch\\
  \hline
  \textbf{Story Points} & 4\\
  \hline
  \textbf{Story}& Als Admin möchte ich Servicemodule erstellen können\\
 \hline
    \textbf{Akzeptanzkriterien} & \\
    \hline
  \textbf{A1} & Das Servicemodul wird erstellt und wird in der Servicemodule Übersicht angezeigt\\
  \hline
\end{tabularx}
 
   \subsubsection{Admin ändert Servicemodul}
\begin{tabularx}{\linewidth}{l X}
  \textbf{Priorität} & Hoch\\
  \hline
  \textbf{Story Points} & 4\\
  \hline
  \textbf{Story}& Als Admin möchte ich Servicemodule ändern können\\
  \hline
   \textbf{Akzeptanzkriterien} & \\
    \hline
  \textbf{A1} & Als Admin krieg ich die Übersicht der verfügbaren Servicemodule\\
  \hline
  \textbf{A2} & Als Admin kann ich das Servicemodule anpassen und speichern\\
  \hline
   \textbf{A3} & Änderungen werden gespeichert\\
 \hline
 \end{tabularx}

 
 \subsubsection{Admin löscht Servicemodul}

 \begin{tabularx}{\linewidth}{l X}
 \textbf{Priorität} & Hoch\\
 \hline
  \textbf{Story Points} & 4\\
  \hline
  \textbf{Story}& Als Admin möchte ich Servicemodule löschen können\\
  \hline
   \textbf{Akzeptanzkriterien} & \\
   \hline
   \textbf{A1} & Als Admin krieg ich die Übersicht der verfügbaren Servicemodule\\
   \hline
   \textbf{A2} & Servicemodule ist gelöscht\\
     \hline
\end{tabularx}

 \subsubsection{Admin geht in die Servicemodule Übersicht}

 \begin{tabularx}{\linewidth}{l X}
\textbf{Priorität} & Hoch\\
 \hline
 \textbf{Story Points} & 2\\
 \hline
 \textbf{Story}& Als Admin möchte ich einen Überblick über die vorhanden Servicemodule angezeigt bekommen\\
 \hline
 \textbf{Akzeptanzkriterien} & \\
 \hline
  \textbf{A1} & Der Admin kann die Servicemodule Übersicht öffnen\\
 \hline
\textbf{A2} & Die Servicemodule Übersicht wird angezeigt\\
  \hline
 \end{tabularx}

\newpage

\section{Nicht-funktionale Anforderungen}

\subsection{Menge}
\begin{itemize}
  \item Die Software unterstützt mindestens 1 Storage Anbieter
  \item Die Software unterstützt mindestens 1 Compute Anbieter
  \item Die Software unterstützt mindestens 1 Network Anbieter
  \item Es soll für Compute, Storage, Network mindestens je 1 Servicemodul erstelt 
  werden
\end{itemize}

\subsection{Schnittstellen}
\begin{itemize}
  \item Die Software wird über HTTP/HTTPS angesprochen
  \item Zur Interaktion im Admin-Dashboard/Customer-Dashboard werden die herkömmlichen 
  Schnittstellen gebraucht (Maus,Tastatur,Bildschirm)
\end{itemize}
\subsection{Qualitätsmerkmale}
\subsubsection{Funktionalität}
siehe Abschnitt API und Dashboard
\subsubsection{Zuverlässigkeit}
\begin{itemize}
  \item Der Workflow zum erstellen eines Services soll entweder durchgeführt und 
  abgeschlossen werden oder falls Unterbruch/Fehler rückgängig gemacht 
  werden.
  \item Die Software soll verteilt betrieben werden und eine möglichst hohe 
  Verfügbarkeit/Zuverlässigkeit bieten
\end{itemize}
\subsubsection{Benutzerbarkeit}
\begin{itemize}
  \item Konfigurationen können über das vorgesehene Admin-Dashboard geändert werden
  \item Zum verwenden der Software besteht noch ein einfaches 
  User-Dashboard
\end{itemize}
\subsubsection{Effizienz}
\begin{itemize}
  \item Die Software Soll mehrere Aufträge von Customern gleichzeitig abarbeiten können
\end{itemize}
\subsubsection{Änderbarkeit}
Die Software soll modular aufgebaut werden, damit Erweiterungen in Zukunft 
problemlos möglich sind.
\subsubsection{Übertragbarkeit}
Das Projekt wird in Java geschrieben und ist somit also auf Java mindestens in der Version 
1.8 angewiesen.


\end{document}