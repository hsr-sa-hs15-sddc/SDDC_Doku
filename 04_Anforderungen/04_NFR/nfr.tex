\subsection{Menge}
\begin{itemize}
  \item Die Software unterstützt mindestens 1 Storage Anbieter
  \item Die Software unterstützt mindestens 1 Compute Anbieter
  \item Die Software unterstützt mindestens 1 Network Anbieter
  \item Es soll für Compute, Storage, Network mindestens je 1 Servicemodul erstellt 
  werden
\end{itemize}

\subsection{Schnittstellen}
\begin{itemize}
  \item Die Software wird über HTTP/HTTPS angesprochen
  \item Zur Interaktion im Admin-Dashboard/Customer-Dashboard werden die herkömmlichen 
  Schnittstellen gebraucht (Maus,Tastatur,Bildschirm)
\end{itemize}
\subsection{Qualitätsmerkmale}
\subsubsection{Funktionalität}
siehe Abschnitt API und Dashboard
\subsubsection{Zuverlässigkeit}
\begin{itemize}
  \item Der Workflow zum erstellen eines Services soll entweder durchgeführt und 
  abgeschlossen werden oder falls Unterbruch/Fehler auftritt rückgängig gemacht
  \item Die Software soll verteilt betrieben werden und eine möglichst hohe 
  Verfügbarkeit/Zuverlässigkeit bieten
\end{itemize}
\subsubsection{Benutzerbarkeit}
\begin{itemize}
  \item Konfigurationen können über das vorgesehene Admin-Dashboard geändert werden
  \item Zum verwenden der Software soll noch ein einfaches 
  User-Dashboard bestehen
\end{itemize}
\subsubsection{Effizienz}
\begin{itemize}
  \item Die Software Soll mehrere Aufträge von Customern gleichzeitig abarbeiten können
\end{itemize}
\subsubsection{Änderbarkeit}
Die Software soll modular aufgebaut werden, damit Erweiterungen in Zukunft möglich sind.
\subsubsection{Übertragbarkeit}
Das Projekt wird in Java geschrieben und ist somit also auf Java mindestens in der Version 
1.8 angewiesen.
