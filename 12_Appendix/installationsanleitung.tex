\chapter{Installationsanleitung}
\section{Einführung}
Für die Software steht ein Docker Container zur Verfügung, welchen es einem sehr 
vereinfacht die Software aufzuschalten.
Dadurch sind die Abhängigkeiten (Java, Libvirt-bin etc.) bereits vorinstalliert 
und der Container muss nur noch gestartet werden.
Der Container, der hier vorgestellt wird verbindet auf die vorgestellte 
Testumgebung, verwendet jedoch eine inMemory Datenbank und verbindet sich auf 
test:///default.

\textbf{Standardport:} 8080
\\
\texbf{Customer Dashboard:} /services
\\
\texbf{Admin Dashboard:} /admin/services

\section{Docker Container}
Der Docker Container ist auf dem Dockerhub aufgeschaltet und kann von dort 
heruntergeladen werden.
\begin{lstlisting}[style=BASH,language=bash,caption={Pull Docker Container}]
#!/binbash
docker pull silvanadrian/sddc_docker
\end{lstlisting}
Oder der Container kann direkt gestartet werden (dann wird er automatisch gepulled)

\begin{lstlisting}[style=Bash,language=bash,caption={Run Docker Container}]
  #!/binbash
  docker run -d -p 80:8080 silvanadrian/sddc_docker
\end{lstlisting}