\documentclass[11pt]{scrartcl}

\title{Projektplan: SDDC}
\author{Silvan Adrian \\ Fabian Binna}
\date{\today{}}

\usepackage[ngerman]{babel}
\usepackage[automark]{scrpage2}
\usepackage{hyperref}
\usepackage{color}
\usepackage[normalem]{ulem}
\usepackage{scrpage2}
\usepackage{graphicx}
\usepackage{tabularx}
\graphicspath{ {./images/} }
\pagestyle{scrheadings}

\clearscrheadfoot
\ihead{\includegraphics[scale=0.3]{SDDC}}
\ohead{Projekt: SDDC}
\ifoot{Projektplan: SDDC}
\cfoot{Version: 1.00}
\ofoot{Datum: \today{}}
\setheadsepline{0.5pt}
\setfootsepline{0.5pt}

\usepackage{ucs}
\usepackage[utf8]{inputenc}
\usepackage[T1]{fontenc}

\begin{document}
\def\arraystretch{1.5}
\begin{titlepage}
\begin{center}
\vspace{10em}
\includegraphics[scale=2]{SDDC}
\vspace{10em}
\end{center}
\begin{center}
\huge {Projektplan}
\end{center}
\begin{center}
\vspace{10em}
\LARGE {Silvan Adrian} \\
\LARGE {Fabian Binna}
\end{center}

\end{titlepage}

\newpage
\section{Änderungshistorie}
\begin{tabularx}{\linewidth}{l l l l}
\textbf{Datum} & \textbf{Version} & \textbf{Änderung}  & \textbf{Autor} \\
\hline
\textbf{17.09.15} & 1.00 & Erstellung des Dokuments & Gruppe \\

\end{tabularx}

\newpage
\tableofcontents
\newpage

\section{Einführung}
\label{sec:Einführung}
\subsection{Zweck}

\subsection{Gültigkeitsbereich}

\subsection{Referenzen}

\section{Projektübersicht}


\subsection{Zweck und Ziel}
 

\subsection{Primäre Features}

\subsection{Erweiterte Features}


\subsection{Lieferumfang}


\subsection{Annahmen und Einschränkungen}

\section{Projektorganisation}

\subsection{Organisationsstruktur}


\subsection{Externe Schnittstellen}



\section{Managment Abläufe}

\subsection{Kostenvoranschlag}


\subsection{Sprints}

\subsection{Besprechungen}



\section{Risikomanagement}

\subsection{Risiken}



\subsection{Umgang mit Risiken}

\section{Arbeitspakete}

\section{Infrastruktur}

\subsection{Entwicklungsinfrastruktur}

\begin{tabularx}{\textwidth}{c c c c}
\textbf{Name} & \textbf{Hardware} & \textbf{Betriebssystem} & \textbf{IDE} \\
\hline
Silvan Adrian & MacBook Pro & OSX 10.10.5 &  \\
\hline
Fabian Binna & Lenovo T430s & Windows 10 &  \\
\hline
\end{tabularx}

\subsection{Tools/Software}

\subsection{Kommunikationsmittel}

\section{Qualitätsmassnahmen}
\subsection{Dokumentation}
Die Dokumentation befindet sich auf einem privaten GitHub Repository.
Die Texte werden in LaTex geschrieben. Die Dokumente werden versioniert.\\
\\
\url{https://github.com/silvanadrian/SDDC_Doku.git}

\subsection{Projektmanagement}
Für das Projektmanagement wird OpenProject verwendet.\\
\\
\url{sddc.silvn.com}


\subsection{Entwicklung}
\subsubsection{Unit Testing / Test-Driven Development}
Die Unit Tests kommen in eine separaten Ordner "Test".
Es wird eine möglichst hohe Code Coverage angestrebt.
Die Code Coverage wird mit einem Tool (z.B. eclEmma) sichergestellt.\\
\\
Die Klassen werden mit Hilfe von Test-Driven Development implementiert.

\subsubsection{Code Review}
Nach jedem Sprint oder bei Abschluss grosser Arbeitspakete wird ein Code Review durchgeführt.\\
\\
Review Protokoll

\subsubsection{Code Style Guidelines}
Editor Standard

\subsection{Testen}

\subsubsection{Komponententest}

\subsection{Systemtest}

\end{document}