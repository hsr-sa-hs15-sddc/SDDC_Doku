\documentclass[11pt]{scrartcl}

\title{Anforderungsspezifikation}
\author{Silvan Adrian \\ Fabian Binna}
\date{\today{}}

\usepackage[ngerman]{babel}
\usepackage[automark]{scrpage2}
\usepackage[colorlinks = true,
linkcolor = black]{hyperref}
\usepackage{color}
\usepackage[normalem]{ulem}
\usepackage{scrpage2}
\usepackage{graphicx}
\usepackage{tabularx}
\graphicspath{ {../22_Grafiken/01_Logo/}{images/}{../../22_Grafiken/01_Logo/} }
\pagestyle{scrheadings}

\clearscrheadfoot
\ihead{\includegraphics[scale=0.3]{SDDC}}
\ohead{Projekt: SDDC}
\ifoot{Template}
\cfoot{Version: 1.00}
\ofoot{Datum: \today{}}
\setheadsepline{0.5pt}
\setfootsepline{0.5pt}

\usepackage{ucs}
\usepackage[utf8]{inputenc}
\usepackage[T1]{fontenc}


\begin{document}
\def\arraystretch{1.5}
\begin{titlepage}
\begin{center}
\vspace{10em}
\includegraphics[scale=2]{SDDC}
\vspace{10em}
\end{center}
\begin{center}
\huge {Anforderungsspezifikation}
\end{center}
\begin{center}
\vspace{10em}
\LARGE {Silvan Adrian} \\
\LARGE {Fabian Binna}
\end{center}

\end{titlepage}

\newpage
\section{Änderungshistorie}
\begin{tabularx}{\linewidth}{l l X l}
\textbf{Datum} & \textbf{Version} & \textbf{Änderung}  & \textbf{Autor} \\
\hline
\textbf{02.10.15} & 1.00 & Erstellung des Dokuments & Gruppe \\
\textbf{02.10.15} & 1.01 & Nicht funktionale Anforderungen & Silvan Adrian\\


\end{tabularx}

\newpage
\tableofcontents
\newpage

\section{Einführung}
\subsection{Zweck}
Dieses Dokument beinhaltet die Anforderung zur Analyse.
\subsection{Gültigkeitsbereich}
Dieses Dokument ist während des ganzen Projekts gültig.


\subsection{Referenzen}
-

\section{Anforderungen}
\subsection{API}


\subsection{Dashboard}


\section{Nichtfunktionale Anforderungen}
\subsection{Menge}
\begin{itemize}
  \item Die Software unterstützt mehr als 30 Cloud Anbieter (libcloud)
  \item Bei jedem Cloud Anbieter bestehen eine gewisse Anzahl Services (von Anbieter zu Anbieter verschieden)
\end{itemize}

\subsection{Schnittstellen}
\begin{itemize}
  \item Die Software wird über HTTP/HTTPS angesprochen
  \item Zur Interaktion im Admin-Dashboard werden die herkömmlichen 
  Schnittstellen gebraucht (Maus,Tastatur,Bildschirm)
  \item Interaktionen können auch über die Kommandozeile ausgeführt werden
\end{itemize}
\subsection{Qualitätsmerkmale}
\subsubsection{Funktionalität}
siehe Abschnitt API und Dashboard
\subsubsection{Zuverlässigkeit}
\begin{itemize}
  \item Der Workflow zum erstellen eines Services soll entweder durchgeführt und 
  abgeschlossen werden oder falls Unterbruch/Fehler rückgängig gemacht 
  werden.
  \item Die Software soll verteilt betrieben werden und eine möglichst hohe 
  Verfügbarkeit bieten
\end{itemize}
\subsubsection{Benutzerbarkeit}
\begin{itemize}
  \item Die Software kann über das vorgesehene Admin-Dashboard benutzt werden
  \item Die API kann auch über die Kommandozeile angesprochen werden
\end{itemize}
\subsubsection{Effizienz}
\begin{itemize}
  \item 
\end{itemize}
\subsubsection{Änderbarkeit}
Die Software soll modular aufgebaut werden, damit Erweiterungen in Zukunft 
problemlos möglich sind.
\subsubsection{Übertragbarkeit}
Das Projekt wird in Python geschrieben ist somit also auf Python mindestens in der Version 2.5 angewiesen, 
kann allerdings durch den Einsatz eines Docker Containers einfach Übertragbar 
gemacht werden.
\end{document}