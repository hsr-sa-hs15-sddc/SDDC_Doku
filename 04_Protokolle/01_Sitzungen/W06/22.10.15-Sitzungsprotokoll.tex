\def\minfileversion{V1.8d}     %^^Aof minutes.sty
\def\minfiledate{2009/12/04}   %^^Aof minutes.sty
%%
%% \CharacterTable
%%  {Upper-case    \A\B\C\D\E\F\G\H\I\J\K\L\M\N\O\P\Q\R\S\T\U\V\W\X\Y\Z
%%   Lower-case    \a\b\c\d\e\f\g\h\i\j\k\l\m\n\o\p\q\r\s\t\u\v\w\x\y\z
%%   Digits        \0\1\2\3\4\5\6\7\8\9
%%   Exclamation   \!     Double quote  \"     Hash (number) \#
%%   Dollar        \$     Percent       \%     Ampersand     \&
%%   Acute accent  \'     Left paren    \(     Right paren   \)
%%   Asterisk      \*     Plus          \+     Comma         \,
%%   Minus         \-     Point         \.     Solidus       \/
%%   Colon         \:     Semicolon     \;     Less than     \<
%%   Equals        \=     Greater than  \>     Question mark \?
%%   Commercial at \@     Left bracket  \[     Backslash     \\
%%   Right bracket \]     Circumflex    \^     Underscore    \_
%%   Grave accent  \`     Left brace    \{     Vertical bar  \|
%%   Right brace   \}     Tilde         \~}
%%
\documentclass{scrreprt}
%%\documentclass{article}
%%\documentclass{scrreprt}
%%\documentclass{scrarctl}
\usepackage[german]{babel}
\usepackage{minutes}
\usepackage[utf8]{inputenc}
\begin{document}
\begin{Protokoll}{Weekly Meeting 22. Oktober}
\moderation{Beat Stettler}
\protokollant{Silvan Adrian, Fabian Binna}
\teilnehmer{Silvan Adrian, Fabian Binna,Beat Stettler}
\fehlendEntschuldigt{Urs Baumann}
\sitzungsdatum{15.\ Oktober 2015}
\sitzungsbeginn{08:25}
\sitzungsende{09:55}
\sitzungsort{Raum C2.101 }
%\verteiler{Vereinsmitglieder}
%\fehlend[entschuldigt]{abwesend}
%%\fehlendEntschuldigt{}
%%\fehlendUnentschuldigt{}
\protokollKopf

\topic{Design/Architektur}%<-- hier Tagesordnungspunkt einfuegen
\subtopic{Generic API}
\begin{itemize}
  \item Generic API Facade noch generischer (createResource/deleteResource)
\end{itemize}

\subtopic{Services verwalten}
\begin{itemize}
  \item Update fehlt momentan noch ganz (gemäss Projektplan oder Sprint nur kündigen und abonnieren)
\end{itemize}

\topic{Abgabe}
\subtopic{Dokument}
\begin{itemize}
  \item Sachlich geschrieben
  \item Ob 1 Dokument oder mehrere spielt keine grosse Rolle
\end{itemize}

\subtopic{Präsentation}
\begin{itemize}
  \item Zum auf oder abrunden (je nach Person und falls auffällt das jemand in einem Bereich weniger Ahnung hat)
\end{itemize}

\topic{Termine und Aufgaben}
\subtopic{Termine}
\termin{2015/10/22}[09:00-09:45]{Projektmeeting im Bau C2.101}

\topic{Aufgaben}
\aufgabe*{Tasks zu User Stories erstellen}
\aufgabe*{Metriken aufsetzen (SonarQube)}
\aufgabe*{Dokumente auf den neusten Stand bringen}
\aufgabe*{Design ausbauen}
\aufgabe*{Grundgerüst aufbauen (Building, Packages, Klassen, Interfaces)}

\end{Protokoll}
\end{document}
\endinput
