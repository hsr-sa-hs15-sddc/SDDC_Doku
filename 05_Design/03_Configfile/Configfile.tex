
\section{Configfile}
Das Configfile beinhaltet die Config für das jeweilige Servicemodul 
(Compute,Storage,Network), dabei gibt es für Storage und Compute 3 Grössen 
(S,M,L), dazu wird dann das nötige Configfile hinterlegt womit der Service 
konfiguriert und aufgesetzt wird.
Mit Ausnahme von Network, wo es keine Grössen gibt.
Die Nachfolgenden Konfigurationen stützen sich hauptsächlich auf Libvirt, bei 
Public Cloud Anbietern wird die Grösse, Betriebssystem, Storage und Netzwerkkonfiguration 
vielmals bereits vorgegeben.

\subsection{Allgemein}
Da je nach Bedarf Compute, Storage oder Netzwerk zuerst erstellt werden muss 
wird dies über einen Workflow behandelt, welcher entscheidet welches
Servicemodul zuerst erstellt werden soll.

\subsection{Compute}
In Compute werden Memory, vCPU, Festplatte und Netzwerk (IP,VLAN) zugewiesen.

\subsubsection{Name}
Jeder Compute Instanz muss einen eindeutigen Namen gegeben werden, dieser wird 
durch den Workflow erstellt.

\subsubsection{Memory}
Memory wird entweder als fester Wert mitgegeben oder wird über die Instanzgrösse 
beim Cloud Anbieter vorbestimmt (micro,medium,large).

\subsubsection{vCPU}
vCPU's werden entweder als Wert mitgegeben oder durch den Cloud Anbieter vorgegeben (die Instanzgrösse). 

\subsubsection{Boot Disk}
Ebenfalls muss noch eine Boot Disk übergeben werden, dies kann ein bereits 
bestehender Storage Pool,Volume oder eine Datei sein (Template mit vorinstallierter Software)
Die Grösse wird hier durch die ausgewählte Disk/Datei bestimmt, bei Cloud 
Anbietern jedoch durch die Instanzgrösse.

\subsubsection{Network}
Beim Netzwerk kann eine IP oder MAC Adresse mitgegeben oder automatisch zugewiesen 
werden, dies geschieht über eine Virtual Bridge oder Eingabe des VLANs.

\subsection{Storage}
\subsubsection{Name}
Storage besitzt einen festen eindeutigen Namen über welchen der Pool angesprochen 
werden kann.
\subsubsection{Grösse}
Der Storage benötigt eine gewisse Grösse um erstellt zu werden oder falls es 
sich um eine Partition oder Datei handelt wird sie dadurch vorgegeben.

\subsubsection{Typ}
Der Storage kann sowohl lokaler als auch Netzwerk Speicher sein, hier wird 
zwischen verschiedenen Typen unterschieden.
Z.B.: können NFS Storages eingebunden werden oder GlusterFS bzw. Sheepdog.

\subsection{Network}
\subsubsection{Name}
Network besitzt einen eindeutigen Namen, welcher nur einmal auf dem System 
vorhanden sein darf.

\subsubsection{IP Family}
Je Nach  Konfiguration muss noch angegeben werden ob IPv4 oder IPv6 Adressen 
verwendet werden sollen.

\subsubsection{IP Range}
Bei der Konfiguration Bspw.: einer Bridge kann auch noch ein IP Range mitgegeben 
werden, welcher an die Angeschlossenen Compute Instanzen an der Bridge verteilt 
werden sollen.
