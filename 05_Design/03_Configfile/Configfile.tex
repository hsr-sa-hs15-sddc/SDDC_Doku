
\section{Configfile}
Das Configfile beinhaltet die Config für das jeweilige Servicemodul 
(Compute,Storage,Network), dabei gibt es für Storage und Compute 3 Grössen 
(S,M,L), dazu wird dann das nötige Configfile ausgewählt womit der Service 
konfiguriert und aufgesetzt wird.
Mit Ausnahme von Network, wo es keine Grössen gibt und  je nach vorgehensweise 
durch einen ``Typ'' (Bridge,Subnetz etc.) oder direkt durch Config hinterlegen gelöst wird.

\subsection{Allgemein}
Da je nach Bedarf Compute, Storage oder Netzwerk zuerst erstellt werden muss 
wird dies über eine Art Workflow behandelt, welcher entscheidet welches
Servicemodul zuerst erstellt werden soll.

\subsection{Compute}
In Compute werden Memory, vCPU, Harddisk und Netzwerk (IP) zugewiesen.
Es gibt noch einige mehr Funktionen, diese werden zu diesem Zeitpunkt aber noch 
nicht behandelt.

\subsubsection{Name/uuid}
Jeder Compute Instanz muss ein eindeutiger Name gegeben werden, dieser wird zu 
beginn direkt in das Configfile eingefügt und zu einem späteren Zeitpunkt soll 
es möglich sein denn Namen via Customer-Dashboard festzulegen.
Bei libvirt wird ebenfalls eine uuid benötigt, welche eine eindeutige 
Identifizierung erlaubt -> wodurch das auffinden und löschen einer Instanz 
erleichtert werden soll

\subsubsection{Memory}
Memory wird entweder als fester Wert mitgegeben oder wird über die Instanzgrösse 
beim Cloud Anbieter vorbestimmt (micro,medium,large).

\subsubsection{vCPU}
vCPU's werden entweder als Wert mitgegeben oder durch den Cloud Anbieter vorgegeben (die Instanzgrösse). 

\subsubsection{Image}
Je nach Anforderung muss der Pfad zu einer Boot ISO mitgegeben werden (für die Installation eines 
Betriebssystem) 

\subsubsection{Boot Disk}
Ebenfalls muss noch eine Boot Disk übergeben werden, dies kann ein bereits 
bestehender Storage Pool,Volume oder eine Datei sein.
Die Grösse wird hier durch die ausgewählte Disk/Datei bestimmt, bei Cloud 
Anbietern jedoch durch die Instanzgrösse.

\subsubsection{Network}
Beim Netzwerk kann eine IP oder MAC Adresse mitgegeben oder automatisch zugewiesen 
werden.

\subsection{Storage}
\subsubsection{Name}
Storage besitzt einen festen eindeutigen Namen über welchen der Pool angesprochen 
werden kann.
\subsubsection{Grösse}
Der Storage benötigt eine gewisse Grösse um erstellt zu werden oder falls es 
sich um eine Partition oder Datei handelt wird sie dadurch vorgegeben.

\subsubsection{Typ}
Der Storage kann sowohl lokaler als auch Netzwerk Speicher sein, hier wird 
zwischen verschiedenen Typen unterschieden.
Z.B.: können NFS Storages eingebunden werden oder GlusterFS bzw. Sheepdog.

\subsection{Network}
\subsubsection{Name}
Network besitzt einen eindeutigen Namen, welcher nur einmal auf dem System 
vorhanden sein darf.

\subsubsection{IP Family}
Je Nach  Konfiguration muss noch angegeben werden ob IPv4 oder IPv6 Adressen 
verwendet werden sollen.

\subsubsection{IP Range}
Bei der Konfiguration Bspw.: einer Bridge kann auch noch ein IP Range mitgegeben 
werden, welcher an die Angeschlossenen Compute Instanzen an der Bridge verteilt 
werden sollen.
