\documentclass[11pt]{scrartcl}

\title{Template}
\author{Silvan Adrian \\ Fabian Binna}
\date{\today{}}

\usepackage[ngerman]{babel}
\usepackage[automark]{scrpage2}
\usepackage[colorlinks = true,
linkcolor = black]{hyperref}
\usepackage{color}
\usepackage[normalem]{ulem}
\usepackage{scrpage2}
\usepackage{graphicx}
\usepackage{tabularx}
\graphicspath{ {../22_Grafiken/01_Logo/}{images/}{../../22_Grafiken/01_Logo/} }
\pagestyle{scrheadings}

\clearscrheadfoot
\ihead{\includegraphics[scale=0.3]{SDDC}}
\ohead{Projekt: SDDC}
\ifoot{Configfile}
\cfoot{Version: 1.02}
\ofoot{Datum: \today{}}
\setheadsepline{0.5pt}
\setfootsepline{0.5pt}

\usepackage{ucs}
\usepackage[utf8]{inputenc}
\usepackage[T1]{fontenc}


\begin{document}
\def\arraystretch{1.5}
\begin{titlepage}
\begin{center}
\vspace{10em}
\includegraphics[scale=2]{SDDC}
\vspace{10em}
\end{center}
\begin{center}
\huge {Configfile}
\end{center}
\begin{center}
\vspace{10em}
\LARGE {Silvan Adrian} \\
\LARGE {Fabian Binna}
\end{center}

\end{titlepage}

\newpage
\section{Änderungshistorie}
\begin{tabularx}{\linewidth}{l l X l}
\textbf{Datum} & \textbf{Version} & \textbf{Änderung}  & \textbf{Autor} \\
\hline
\textbf{17.10.15} & 1.00 & Erstellung des Dokuments & Gruppe \\
\textbf{19.10.15} & 1.01 & Configfile Optionen& Silvan Adrian \\
\textbf{07.11.15} & 1.02 & JSON definition + Anpassungen auf neue Anforderungen & Fabian Binna\\
\end{tabularx}

\newpage
\tableofcontents
\newpage
\section{Einführung}
\subsection{Zweck}
Dieses Dokument beschreibt das Configfile
\subsection{Gültigkeit}
Dieses Dokument ist währen des ganzen Projekts gültig und wird laufend erweitert
\subsection{Übersicht}
Dieses Dokument soll eine Übersicht über die Inhalte des Configfiles geben.

\section{Configfile}
Das Configfile beschreibt einen Service möglichst abstakt. Die Punkte Compute und Storage
werden durch allgemeingültige Grössen (S, M, L) beschrieben. Der eigentliche Service ist auf einem
Image vorinstalliert. Im Configfile wird dieses Image referenziert und bei aufsetzten des Services auf
dem Server installiert.

\subsection{Allgemein}
Da je nach Bedarf Compute, Storage oder Netzwerk zuerst erstellt werden muss 
wird dies über eine Art Workflow behandelt, welcher entscheidet welches
Servicemodul zuerst erstellt werden soll.

\subsubsection{JSON}
Das Configfile wird in JSON geschrieben und wird auch so an die Generische API 
übergeben.

\subsection{Compute}
In Compute werden Memory, vCPU, Harddisk und Netzwerk (IP) zugewiesen.
Es gibt noch einige mehr Funktionen, diese werden zu diesem Zeitpunkt aber noch 
nicht behandelt.

\subsubsection{Name/uuid}
Jeder Compute Instanz muss ein eindeutiger Name gegeben werden, dieser wird zu 
beginn direkt in das Configfile eingefügt und zu einem späteren Zeitpunkt soll 
es möglich sein denn Namen via Customer-Dashboard festzulegen.
Bei libvirt wird ebenfalls eine uuid benötigt, welche eine eindeutige 
Identifizierung erlaubt -> wodurch das auffinden und löschen einer Instanz 
erleichtert werden soll

\subsubsection{Memory}
Memory wird entweder als fester Wert mitgegeben oder wird über die Instanzgrösse 
beim Cloud Anbieter vorbestimmt (micro,medium,large).

\subsubsection{vCPU}
vCPU's werden entweder als Wert mitgegeben oder durch den Cloud Anbieter vorgegeben (die Instanzgrösse). 

\subsubsection{Boot ISO}
Je nach Anforderung muss der Pfad zu einer Boot ISO mitgegeben werden (für die Installation eines 
Betriebssystem) 

\subsubsection{Boot Disk}
Ebenfalls muss noch eine Boot Disk übergeben werden, dies kann ein bereits 
bestehender Storage Pool,Volume oder eine Datei sein.
Die Grösse wird hier durch die ausgewählte Disk/Datei bestimmt, bei Cloud 
Anbietern jedoch durch die Instanzgrösse.

\subsubsection{Network}
Beim Netzwerk kann eine IP oder MAC Adresse mitgegeben oder automatisch zugewiesen 
werden.
Weitere Netzwerk Einstellungen werden über die SDN Library abgehandelt.
Bei Cloud Anbietern wird hier ebenfalls direkt automatisch eine IP und MAC 
Adresse zugewiesen.

\subsection{Storage}
\subsubsection{Name}
Storage besitzt einen festen eindeutigen Namen über welchen der Pool angesprochen 
werden kann.
\subsubsection{Grösse}
Der Storage benötigt eine gewisse Grösse um erstellt zu werden oder falls es 
sich um eine Partition oder Datei handelt wird sie dadurch vorgegeben.

\subsubsection{Typ}
Der Storage kann sowohl lokaler als auch Netzwerk Speicher sein, hier wird 
zwischen verschiedenen Typen unterschieden.
Z.B.: können NFS Storages eingebunden werden oder GlusterFS bzw. Sheepdog.



\subsection{Network}
\subsubsection{OpenFlow}
Netzwerkeinstellungen werden über OpenFlow gehandhabt, somit werden hier die 
OpenFlow spezifische Konfigurationen unterstützt.
Zu diesem Zeitpunkt wird dies allerdings noch nicht behandelt.


\subsection{Syntax}

\subsubsection{Service}

\{\\
id : <number>,\\
serviceName : <string>,\\
image : <string>,\\

compute : \{\\
size : S | M | L\\
\},\\

storage : \{\\
size : S | M | L\\
\},\\
\}\\

\subsubsection{OrderedService}

\{\\
id : <number>,\\
serviceName : <string>,\\

identifiers : [ <string>, ... ]\\
\}\\


\end{document}