\section{Ist-Situation}
\subsection{API}
Es gibt einige Beispiele von generischen APIs, welche zwar mehrere Anbieter 
ansteuern können.
Jedoch wird meistens zwischen Public und Private Cloud Anbietern APIs 
unterschieden, wodurch keine Hybrid Services möglich sind.
Zudem werden bei Public Cloud Anbietern vielmals nur zwischen Storage und 
Compute unterschieden und Network Services sind eher selten.
Ausnahmen sind hier Clouds DNS oder ein Cloud Loadbalancer.
Für die Private Clouds hat sich bisher OpenStack und CloudStack durchgesetzt, 
diese sind wiederum aber mehr auf Private Clouds ausgerichtet und nicht auf Hybrid 
Clouds.
Ebenfalls ist es nicht gerade einfach OpenStack oder CloudStack im Businees 
Bereich aufzusetzen und den Aufwand ist für manche KMUs zu aufwendig.



\subsection{Dashboard}

Im Dashboard Bereich gibt es nicht gerade viele Angebote, ausser gerade Bitnami.
Bitnami erlaubt es einem seine Cloud Instanzen zu managen an einem Ort, wodurch 
das handhaben von vielen Servern einiges vereinfacht wird.
Jedoch fehlt hier völlig eine Unterstützung für Private Clouds und bietet auch keine Self Hosted Lösung an.