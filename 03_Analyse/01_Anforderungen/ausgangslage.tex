\section{Ist-Situation}
\subsection{API}
Es gibt einige Beispiele von generischen APIs, welche zwar mehrere Anbieter 
ansteuern können, jedoch wird meistens zwischen Public und Private Cloud Anbietern APIs 
unterschieden wodurch keine Hybrid Services möglich sind.
Zudem wird bei Public Cloud Anbietern vielmals nur zwischen Storage und 
Compute unterschieden und Network Service Angebote sind eher selten.
Ausnahmen sind hier Clouds DNS oder ein Cloud Loadbalancer.
Für die Private Clouds hat sich bisher OpenStack und CloudStack durchgesetzt, 
daher sind Hybrid Services nicht einfach so möglich.
Ebenfalls ist es nicht gerade einfach OpenStack oder CloudStack
 aufzusetzen und den Aufwand ist für manche KMUs zu aufwendig oder zu teuer.



\subsection{Dashboard}

Im Dashboard Bereich gibt es nicht gerade viele Angebote, einer der bekanntesten ist Bitnami.
Bitnami erlaubt es einem seine Cloud Instanzen an einem Ort zu managen, wodurch 
das handhaben von vielen verschiedenen Instanzen einiges vereinfacht wird.
Jedoch fehlt hier völlig die Unterstützung für Private Clouds und bietet auch keine Self Hosted Lösung an, 
um die eigene Umgebung einzubauen.