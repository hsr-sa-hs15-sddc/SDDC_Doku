%Alle Referenzen ins Bib File verschieben
\subsection{Referenzen}
\href{https://libcloud.apache.org}{Libcloud}\\
\href{https://jclouds.apache.org}{jClouds}\\
\href{https://github.com/esl/elibcloud}{elibcloud}\\
\href{https://github.com/fog/fog/blob/master/lib/fog/openstack/docs/getting_started.md}{fog}\\
\href{https://github.com/pkgcloud/pkgcloud}{pkgcloud}\\
\href{http://http://libvirt.org/}{libvirt}\\

\newpage

\section{API Analyse}
\subsection{\href{https://libcloud.apache.org}{Libcloud}}
\textbf{Sprache: }Python\\
\textbf{Wichtigste Provider: }Rackspace, Amazon web services, CloudStack, OpenStack, DigitalOcean, Eucalyptus, Joyent, Linode, exoscale,NephoScale, Google Cloud Platform, Zerigo, CloudSigma, iKoula, libvirt\\

\subsection{\href{https://jclouds.apache.org}{jClouds}}
\textbf{Sprache: }Java\\
\textbf{Wichtigste Provider: }OpenStack, Docker, DigitalOcean, Google Cloud Platform, 
Rackspace, HP Cloud, CloudStack, Amazon web services, abiquo, CloudSigma, joyent\\

\subsection{\href{https://github.com/esl/elibcloud}{elibcloud}}
\textbf{Sprache: }Erlang\\
elibcloud ist ein Wrapper für libcloud.\\

\subsection{\href{https://github.com/fog/fog/blob/master/lib/fog/openstack/docs/getting_started.md}{fog}}
\textbf{Sprache: }Ruby\\
\textbf{Wichtigste Provider: }CloudSigma, CloudStack, GoGrid, Google Cloud Platform, Joyent, 
Libvirt, Linode, OpenStack, OpenVZ, Rackspace, Zerigo, IBM, HP\\

\subsection{\href{https://github.com/pkgcloud/pkgcloud}{pkgcloud}}
\textbf{Sprache: }JavaScript (Node.js)\\
\textbf{Wichtigste Provider: }Amazon, Azure, DigitalOcean, Joyent, OpenStack, Rackspace, Google, HP\\

\subsection{\href{http://http://libvirt.org/}{libvirt}}
\textbf{Sprache: }C\\
\textbf{Wichtigste Provider: }Xen, KVM, OpenVZ, VMware ESX, VirtualBox, IBM PowerVM\\

\newpage

\section{Support}
\subsection{Compute}
Die grösste Auswahl an Providern liefert Libcloud.JClouds hingegen 
unterstützt auch Docker, was ein grosser Vorteil gegenüber Libcloud ist. Im 
Dokument Compute.ods im Ordner 02\_Analyse\//01\_API wird genau aufgeführt, 
welche Provider von welchen APIs unterstützt werden. Es werden nur public Clouds berücksichtigt.
Für den private Cloud Bereich bietet sich hier eher Libvirt an, da neben XEN, 
KVM, Qemu und weitere unterstützt werden.

\subsection{Storage (Object/Blob)}
\textbf{libcloud}
\begin{itemize}
\item PCextreme AuroraObjects
\item Microsoft Azure (blobs)
\item CloudFiles
\item Google Storage
\item KTUCloud Storage
\item Numbus.io
\item Ninefold
\item OpenStack Swift
\item Amazon
\end{itemize}

\textbf{jclouds (BlobStore)}
\begin{itemize}
\item AWS
\item HP Helion
\item Azure
\item Rackspace
\end{itemize}

\textbf{fog}
\begin{itemize}
\item S3
\item CloudFiles
\item Google Storage
\end{itemize}

\textbf{pkgcloud}
\begin{itemize}
\item Amazon
\item Azure
\item Google
\item HP
\item OpenStack
\item Rackspace
\end{itemize}

\textbf{libvirt}
\begin{itemize}
  \item GlusterFS
  \item Sheepdog
  \item SCSI
  \item iSCSI
  \item FiberChannel
  \item NFS
  \item lvm
  \item filesystems
\end{itemize}

\subsection{Network}
Libvirt allein bietet hier mit Abstand die beste Unterstützung von Network Konfigurationen 
(VLANs, Bridges, etc.) und ist sehr auf Private Clouds ausgelegt.\\


\subsection{Other}
\subsubsection{Database}
\textbf{pkgcloud}
\begin{itemize}
\item IrisCouch
\item MongoLab
\item Rackspace
\item MongoHQ
\item RedisToGo
\end{itemize}

\subsubsection{DNS}
\textbf{libcloud}
\begin{itemize}
\item AuroraDNS
\item DigitalOcean
\item Gandi
\item Google
\item Host Virtual
\item Linode
\item Rackspace
\item AWS Route53
\item Softlayer
\item Zerigo
\end{itemize}

\textbf{fog}
\begin{itemize}
\item AWS Route53
\item Blue Box
\item DNSimple
\item Linode
\item Rackspace
\item Rage4
\item Slicehost
\item Zerigo
\end{itemize}

\textbf{pkgcloud}
\begin{itemize}
\item Rackspace
\end{itemize}

\subsubsection{Load Balancer}
\textbf{libcloud}
\begin{itemize}
\item Brightbox
\item CloudStack
\item DimensionData
\item Amazon
\item Google
\item GoGrid
\item Ninefold
\item Rackspace
\item Softlayer
\end{itemize}

\textbf{jclouds}
\begin{itemize}
\item AWS Elastic LoadBalancer
\item Rackspace
\end{itemize}

\textbf{pkgcloud}
\begin{itemize}
\item Rackspace
\end{itemize}

\subsubsection{Orchestration}
\textbf{pkgcloud (beta)}
\begin{itemize}
\item OpenStack
\item Rackspace
\end{itemize}

\subsubsection{CDN}
\textbf{fog}
\begin{itemize}
\item CloudFront
\end{itemize}

\newpage

\section{Fazit}

Im Gesamtbild schneidet libcloud am besten ab. Es bietet deutlich am meisten Compute und 
Storage Provider. Die Dokumentation ist sehr ausführlich, mit konkreten Ratschlägen zur 
Implementation (z.B. Thread Safe). Zusätzlich bietet libcloud Module für SSL und Pricing.
Jclouds ist eine Library für Java, was für uns am besten ist, da wir am meisten Erfahrung mit 
Java haben. Es gibt jedoch nicht viele Compute Provider, dafür unterstützt jclouds als einziger Docker.
Der einzige Vorteil von fog ist die Möglichkeit CDNs als Service anzubieten.
Pkgcloud unterstützt eine breite Auswahl von Services (z.B. Database, Load Balancer, DNS).
Elibcloud ist ein erlang Wrapper für libcloud und unterstützt somit das gleiche wie libcloud 
(sonlange die Version auf dem neusten stand ist). Erlang würde sich für eine parallele Umgebung eignen, 
es sind jedoch keinerlei Erlang Kenntnisse im Team vorhanden.\\
Libvirt ist allerdings die einzige API, die sich völlig auf Private Anbieter 
zugeschnitten ist (wird auch von OpenStack verwendet), wäre also die Beste 
Lösung, wenn es darum geht eine Private Cloud aufzubauen.

\textbf{Wir entscheiden uns für libvirt. Die Gründe dafür sind unten im API Matrixvergleich aufgeführt}


\subsection{libcloud}
\begin{Argumentation}
\pro Grösste Auswahl an Compute und Storage Provider.
\pro Am besten dokumentiert. Für jede Methode existiert eine Tabelle, die zeigt welche Provider damit angesprochen werden können.
\pro Ist zwar nicht Thread-Safe. Es werden jedoch konkrete Lösungsvorschläge gemacht.
\pro SSL und Pricing Module vorhanden.

\contra Team hat wenig Erfahrung mit komplexen/grossen Python Projekten.
\end{Argumentation}

\subsection{jclouds}
\begin{Argumentation}
\pro Unterstützt Docker.
\pro Java Library. Das Team hat am meisten Erfahrung mit Java.
\pro Code Examples für fast jeden Provider.

\contra Kleine Auswahl an Compute Providern.
\end{Argumentation}

\subsection{fog}
\begin{Argumentation}
\pro Es ist möglich ein CDN als Service anzubieten.

\contra Mässige Dokumentation. Es existieren zwar 
Examples, die sind aber nicht besonders aussagekräftig.
\contra Kleine Auswahl an Compute Providern.
\end{Argumentation}

\subsection{pkgcloud}
\begin{Argumentation}
\pro Grösste Auswahl an Services.
\pro Database as a Service
\pro Orchestration
\pro Explizite Unterstützung von Network.

\contra Mässige Dokumentation. Es existieren zwar Examples, die sind aber nicht besonders aussagekräftig.
\contra Kleine Auswahl an Compute Providern.
\end{Argumentation}

\subsection{libvirt}
\begin{Argumentation}
\pro Grösste Auswahl an ``Private Cloud Anbietern''.
\pro Grosse Storage Unterstützung.
\pro Explizite Unterstützung von Network.
\pro Bietet Java Library (language Bindings)

\contra Ungenügende Dokumentation. Es existieren zwar Examples und 
sonst existiert nur eine Javadoc zur Library.
\end{Argumentation}









