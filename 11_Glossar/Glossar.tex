\documentclass[11pt]{scrartcl}

\title{Glossar}
\author{Silvan Adrian \\ Fabian Binna}
\date{\today{}}

\usepackage[ngerman]{babel}
\usepackage[automark]{scrpage2}
\usepackage[colorlinks = true,
linkcolor = black]{hyperref}
\usepackage{color}
\usepackage[normalem]{ulem}
\usepackage{scrpage2}
\usepackage{graphicx}
\usepackage{tabularx}
\graphicspath{ {../22_Grafiken/01_Logo/}{images/}{../../22_Grafiken/01_Logo/} }
\pagestyle{scrheadings}

\clearscrheadfoot
\ihead{\includegraphics[scale=0.3]{SDDC}}
\ohead{Projekt: SDDC}
\ifoot{Glossar}
\cfoot{Version: 1.02}
\ofoot{Datum: \today{}}
\setheadsepline{0.5pt}
\setfootsepline{0.5pt}

\usepackage{ucs}
\usepackage[utf8]{inputenc}
\usepackage[T1]{fontenc}


\begin{document}
\def\arraystretch{1.5}
\begin{titlepage}
\begin{center}
\vspace{10em}
\includegraphics[scale=2]{SDDC}
\vspace{10em}
\end{center}
\begin{center}
\huge {Glossar}
\end{center}
\begin{center}
\vspace{10em}
\LARGE {Silvan Adrian} \\
\LARGE {Fabian Binna}
\end{center}

\end{titlepage}

\newpage
\section{Änderungshistorie}
\begin{tabularx}{\linewidth}{l l X l}
\textbf{Datum} & \textbf{Version} & \textbf{Änderung}  & \textbf{Autor} \\
\hline
\textbf{08.10.15} & 1.00 & Erstellung des Dokuments & Gruppe \\
\textbf{08.10.15} & 1.01 & Erste Begriffe und Abkürzungen erklärt & Silvan Adrian\\
\textbf{17.10.15} & 1.02 & Libvirt eingefügt und Fehlerbehebungen & Silvan Adrian\\
\end{tabularx}

\newpage
\tableofcontents
\newpage

\section{Glossar}
\subsection{Begriffe}

\begin{tabularx}{\linewidth}{l | X}
    \textbf{Begriff} & \textbf{Beschreibung}\\
    \hline
     Slack &  Slack ist eine Software, die Kommunikation im Team und Notifications von 
     Commits oder Builds erlauben\\
    \hline
    libcloud & Eine Libary, die in Python geshrieben ist und ein generisches 
    Interface zur Verfügung stellt um mehr als 30 Cloud Anbieter anzusprechen.\\
    \hline
    Bitnami & Ein Dashboard Anbieter um vorkonfigurierte Applikationen schnell in die Cloud 
    stellen zu können.\\
    \hline
    libvirt & Eine Libary, die viele bekannte Hypervisor unterstützt und deren 
    Konfiguration zulässt\\
    \hline
\end{tabularx}

\subsection{Abkürzungen}
\begin{tabularx}{\linewidth}{l | X}
    \textbf{Abkürzung} & \textbf{Beschreibung}\\
    \hline
    SDDC & Software Defined Data Center\\
    \hline
    SDN & Software Defined Network\\
    \hline
    SDS & Software Defined Storage\\
    \hline
    SDC & Software Defined Computation\\
    \hline
    API & Application Programming Interface (Programmierschnittstelle)\\
    \hline
    IaaS & Infrastructure as a Service\\
    \hline
    PaaS & Platfrom as a Service\\
    \hline
    SaaS & Software as a Service\\
\end{tabularx}



\end{document}